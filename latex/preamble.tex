%%%%%%%%%%%%%%%%%%%%%%%%%%%%%
% Loading relevant packages %
%%%%%%%%%%%%%%%%%%%%%%%%%%%%%

%\usepackage[T1]{fontenc} % special characters with pdflatex
%\usepackage[utf8]{inputenc} % special characterswith pdflatex

\usepackage{fontspec} % special characters with xelatex

\usepackage{tikz}
\usepackage{lscape}
\usepackage{indentfirst}
\usepackage{float}
\usepackage[flushleft]{threeparttable}
\usepackage[fulladjust]{marginnote}
\usepackage{tcolorbox}
\usepackage{pdfpages}
\usetikzlibrary{intersections}
\tcbuselibrary{listings,breakable}
\usepackage{pifont}
\usepackage{hyperref}
\usepackage{graphicx,pdflscape}
\usepackage{geometry}
\usepackage{float}
\usepackage{longtable}
\usepackage{supertabular}
\usepackage{subfig}
\usepackage{scrextend}
\usepackage{tabularx}
\usepackage{lscape}
\usepackage{tabu}
\usepackage{array}
\usepackage[gen]{eurosym}
\usepackage{subfig}
\usepackage{stackrel,amssymb}
\usepackage{textcomp}
\usepackage{setspace}

\usepackage{microtype} % make better looking pdf
\usepackage{booktabs,caption,fixltx2e}
\usepackage[none]{hyphenat} % remove hyphenation

\usepackage[a4paper]{./cover/cover_page} % specifies the path to the cover page template

%%%%%%%%%%%%%%%%%%%%%%%%%%%%%%%%%%%%%%%%%%%%%%%%%%%%%%%%%%%%%%%%%%%%%
% Below is the by-default configuration of bookdown-demo            %
% https://github.com/rstudio/bookdown-demo/blob/master/preamble.tex %
%%%%%%%%%%%%%%%%%%%%%%%%%%%%%%%%%%%%%%%%%%%%%%%%%%%%%%%%%%%%%%%%%%%%%

\usepackage{graphicx}
\usepackage{amsthm}

\makeatletter

\def\thm@space@setup{
  \thm@preskip=8pt plus 2pt minus 4pt
  \thm@postskip=\thm@preskip
}

\makeatother

%%%%%%%%%%%%%%%%%%%%%%%%
% Remove default title %
% https://stackoverflow.com/questions/45963505/coverpage-and-copyright-notice-before-title-in-r-bookdown
%%%%%%%%%%%%%%%%%%%%%%%%

\let\oldmaketitle\maketitle
\AtBeginDocument{\let\maketitle\relax}

%%%%%%%%%%%%%%%%%%%%%%%%%%%%%%%%%%%%%%%%%%%%%%%%%%%%%%%%%%%%%
% Add a lettrine to the very first character of the content %
%%%%%%%%%%%%%%%%%%%%%%%%%%%%%%%%%%%%%%%%%%%%%%%%%%%%%%%%%%%%M

\usepackage{lettrine}

\newcommand{\initial}[1]{
	\lettrine[lines=3,lhang=0.33,nindent=0em]{
		\color{gray}
     		{\textsc{#1}}}{}}

%%%%%%%%%%%%%%%%%%%
% Font and format %
%%%%%%%%%%%%%%%%%%%

% Font with math support: New Century Schoolbook
% \usepackage{fouriernc}
% \usepackage[T1]{fontenc}

%%%%%%%%%%%%%%%%%%%%%%%%%%%%%%%%%%%%%%%%%%%%%%%%%%%%%%%%%%%%%
% https://texdoc.net/texmf-dist/doc/latex/memoir/memman.pdf %
%%%%%%%%%%%%%%%%%%%%%%%%%%%%%%%%%%%%%%%%%%%%%%%%%%%%%%%%%%%%%

%\setstocksize{11in}{8.5in} % "real" paper size, already set by a4paper class option
%\settrimmedsize{11in}{8.5in}{*} % trimmed paper size (trimmed on the left and right)

% Spine and trim page margins from main typeblock
%\setlrmarginsandblock{25mm}{25mm}{*}

% Top and bottom page margins from main typeblock
%\setulmarginsandblock{15mm}{20mm}{*}

%\setlrmargins{25mm}{25mm}{*}
%\setulmargins{1.0in}{*}{*}

%\setheadfoot{13pt}{26pt}
%\setheaderspaces{*}{13pt}{*} % set space between header and block

% apply and enforce the layout
%\checkandfixthelayout

% ensure single spacing
\frenchspacing

%%%%%%%%%%%%%%%%%%%%%%%%%%%%%%%%%%%%%%%%%%%%%%%%%%%%%%%%%%%%%%%%%%%%%%%%%%%%%%%
% Chapter style (taken and slightly modified from Lars Madsen Memoir Chapter) %
%%%%%%%%%%%%%%%%%%%%%%%%%%%%%%%%%%%%%%%%%%%%%%%%%%%%%%%%%%%%%%%%%%%%%%%%%%%%%%%

\usepackage{calc,soul,fourier}

\makeatletter
\newlength\dlf@normtxtw
\setlength\dlf@normtxtw{\textwidth}
\newsavebox{\feline@chapter}
\newcommand\feline@chapter@marker[1][4cm]{%
	\sbox\feline@chapter{%
		\resizebox{!}{#1}{\fboxsep=1pt%
			\colorbox{gray}{\color{white}\thechapter}%
		}}%
		\rotatebox{90}{%
			\resizebox{%
				\heightof{\usebox{\feline@chapter}}+\depthof{\usebox{\feline@chapter}}}%
			{!}{\scshape\so\@chapapp}}\quad%
		\raisebox{\depthof{\usebox{\feline@chapter}}}{\usebox{\feline@chapter}}%
}

\newcommand\feline@chm[1][4cm]{%
	\sbox\feline@chapter{\feline@chapter@marker[#1]}%
	\makebox[0pt][c]{% aka \rlap
		\makebox[1cm][r]{\usebox\feline@chapter}%
	}}

\makechapterstyle{daleifmodif}{
\renewcommand\chapnamefont{\normalfont\Large\scshape\raggedleft\so}
\renewcommand\chaptitlefont{\normalfont\Large\bfseries\scshape}
\renewcommand\chapternamenum{} \renewcommand\printchaptername{}
\renewcommand\printchapternum{\null\hfill\feline@chm[2.5cm]\par}
\renewcommand\afterchapternum{\par\vskip\midchapskip}
\renewcommand\printchaptertitle[1]{\color{gray}\chaptitlefont\raggedleft
  \renewcommand\chaptername{Chapter}
  ##1\par}
}

\makeatother
\chapterstyle{daleifmodif}

% The pages should be numbered consecutively at the bottom centre of the page
\makepagestyle{myvf}
\makeoddfoot{myvf}{}{\thepage}{}
\makeevenfoot{myvf}{}{\thepage}{}
\makeheadrule{myvf}{\textwidth}{\normalrulethickness}
\makeevenhead{myvf}{\small\textsc{\leftmark}}{}{}
\makeoddhead{myvf}{}{}{\small\textsc{\rightmark}}
\pagestyle{myvf}

%%%%%%%%%%%%%%%%%%%%%%%%%%%%%%%%%%%%%%%%%%%%%%%%%%%%%%%%
% Create summary box to put at the end of each chapter %
%%%%%%%%%%%%%%%%%%%%%%%%%%%%%%%%%%%%%%%%%%%%%%%%%%%%%%%%

\newtcolorbox[]{summary}[2][]{title=#2,#1}

%%%%%%%%%%%%%%%%%%%%%%%%%%%%%%%%%%%%%%
% Create explanation Box environment %
%%%%%%%%%%%%%%%%%%%%%%%%%%%%%%%%%%%%%%

\tcbuselibrary{skins,breakable}
\usepackage{lipsum}

\newtcolorbox[auto counter, number within = chapter, number freestyle = {\noexpand\thechapter.\noexpand\arabic{\tcbcounter}}]{mybox}[2][]{%
    enhanced,
    %breakable,
    fonttitle = \bfseries,
    title = Box~\thetcbcounter: #2,
    #1
}

%%%%%%%%%%%%%%%%%%%%%%%%%%%%%%%%%%%%%
% Get the number of current chapter %
%%%%%%%%%%%%%%%%%%%%%%%%%%%%%%%%%%%%%

\newcommand\getcurrentref[1]{
 \ifnumequal{\value{#1}}{0}
  {??}
  {\the\value{#1}}
}

%%%%%%%%%%%%%%%%%%%%%%%%
% Insert an empty page %
%%%%%%%%%%%%%%%%%%%%%%%%

\usepackage{afterpage}

\newcommand\blankpage{%
    \null
    \thispagestyle{empty}%
    % \addtocounter{page}{-1}% % uncomment to increase page counter
    \newpage
    }

%%%%%%%%%%%%%%%%%%%%%
% Insert a timeline %
%%%%%%%%%%%%%%%%%%%%%

% \usepackage{chronology}

%%%%%%%%%%%%%%%%%%
% Epigraph style %
%%%%%%%%%%%%%%%%%%

\usepackage{epigraph}
\setlength\epigraphwidth{1\textwidth}
\setlength\epigraphrule{0pt} % no line between
\setlength\beforeepigraphskip{1\baselineskip} % space before and after epigraph
\setlength\afterepigraphskip{2\baselineskip}
\renewcommand*{\textflush}{flushright}
\renewcommand*{\epigraphsize}{\normalsize\itshape}
