%%%%%%%%%%%%%%%%%%%%%%%%%%%%%
% Loading relevant packages %
%%%%%%%%%%%%%%%%%%%%%%%%%%%%%

%\usepackage[latin1]{inputenc} % special characters with pdflatex
%\usepackage[utf8]{inputenc} % special characters with pdflatex
%\inputencoding{latin1} % special characters with pdflatex

\usepackage{fontspec} % special characters (e.g., accents) with xelatex or luatex
\usepackage{pdflscape} % figures in landscape mode
\usepackage{tikz} % creating graphic elements
\usepackage{lscape} % figures or tables in landscape mode
\usepackage{indentfirst} % indents first pragraph after section header
\usepackage{float} % improves the interface for defining floating objects (e.g., figures or tables)
\usepackage[flushleft]{threeparttable} % allows for three part tables with a specified notes section
\usepackage[fulladjust]{marginnote} % provides the \marginnote command
\usepackage{tcolorbox} % provides environment for colored boxes
\usepackage{pdfpages} % simplifies the inclusion of PDF documents

\usetikzlibrary{intersections} % used to determine lines intersections
\tcbuselibrary{listings,breakable}

\usepackage{pifont} % provides commands for pi fonts (e.g., dingbats, symbol)
\usepackage{graphicx} % provides optionnal arguments to the \includegraphics command
\usepackage{geometry} % provides an easy and flexible user interface to customize page layout
\usepackage{longtable} % allows writing tables that extend on several pages
\usepackage{supertabular} % the predecessor of the longtable package
\usepackage{scrextend} % uses components of KOMA-Script
\usepackage{tabularx} % defines the tabularx environment
\usepackage{lscape} % modifies the margins and rotates the page contents but not the page number.
\usepackage{tabu} % provides flexible latex table environment (e.g., long table)
\usepackage{array} % an extended implementation of the array and tabular environments
\usepackage[gen]{eurosym} % provides the euro (€) symbol
\usepackage{gensymb} % provides the degree (°) symbol
\usepackage{subfig} % provides support for the manipulation and reference of small or ‘sub’ figures and tables within a single figure or table environment
\usepackage{stackrel,amssymb}
\usepackage{textcomp} % provides many text symbols
\usepackage{setspace} % provides support for setting the spacing between lines
\usepackage{microtype} % makes better looking pdf
\usepackage{booktabs,caption,fixltx2e} % ...
\usepackage[none]{hyphenat} % removes hyphenation
\usepackage[a4paper]{./cover/cover_page} % specifies the path to the cover page template

%%%%%%%%%%%%%%%%%%%%%%%%%%%%%%%%%%%%%%%%%%%%%%%%%%%%%%%%%%%%%%%%%%%%%
% Below is the by-default configuration of bookdown-demo            %
% https://github.com/rstudio/bookdown-demo/blob/master/preamble.tex %
%%%%%%%%%%%%%%%%%%%%%%%%%%%%%%%%%%%%%%%%%%%%%%%%%%%%%%%%%%%%%%%%%%%%%

%%%%%%%%%%%%%%%%%%%%%%%%
% Remove default title %
% https://stackoverflow.com/questions/45963505/coverpage-and-copyright-notice-before-title-in-r-bookdown
%%%%%%%%%%%%%%%%%%%%%%%%

\let\oldmaketitle\maketitle
\AtBeginDocument{\let\maketitle\relax}

%%%%%%%%%%%%%%%%%%%%%%%%%%%%%%%%%%%%%%%%%%%%%%%%%%%%%%%%%%%%%
% Add a lettrine to the very first character of the content %
%%%%%%%%%%%%%%%%%%%%%%%%%%%%%%%%%%%%%%%%%%%%%%%%%%%%%%%%%%%%M

\usepackage{lettrine} % supports various dropped capitals styles

\newcommand{\initial}[1]{
	\lettrine[lines=3,lhang=0.33,nindent=0em]{
		\color{gray}
     		{\textsc{#1}}}{}}

%%%%%%%%%%%%%%%%%%%%%%%%%%%%%%%%%%%%%%%%%%%%%%%%%%%%%%%%%%%%%
% Format of the page (block, header, margins, etc.)         %
% https://texdoc.net/texmf-dist/doc/latex/memoir/memman.pdf %
%%%%%%%%%%%%%%%%%%%%%%%%%%%%%%%%%%%%%%%%%%%%%%%%%%%%%%%%%%%%%

% Sets the "real" paper size (already set by a4paper class option)
% \setstocksize{11in}{8.5in}

% Trimmed paper size (trimmed on the left and right)
% \settrimmedsize{11in}{8.5in}{*}

% Sets the \headheight and \footskip parameters (respectively)
\setheadfoot{\onelineskip}{2\onelineskip}

% Sets space between header and block
\setheaderspaces{*}{2\onelineskip}{*}

% Spine and trim page margins from main typeblock (left and right margins if oneside)
\setlrmarginsandblock{25mm}{25mm}{*}

% Top and bottom page margins from main typeblock
\setulmarginsandblock{25mm}{*}{1}

% Applies and enforces the layout
\checkandfixthelayout

% Ensures single spacing between sentences
\frenchspacing

%%%%%%%%%%%%%%%%%%%%%%%%%%%%%%%%%%%%%%%%%%%%%%%%%%%%%%%%%%%%%%%%%%%%%%%%%%%%%%%
%%%%%%%%%%%%%%%%%%%%%%%%%%%%%%%%%%%%%%%%%%%%%%%%%%%%%%%%%%%%%%%%%%%%%%%%%%%%%%%
% Chapter style (taken and slightly modified from Lars Madsen Memoir Chapter) %
%%%%%%%%%%%%%%%%%%%%%%%%%%%%%%%%%%%%%%%%%%%%%%%%%%%%%%%%%%%%%%%%%%%%%%%%%%%%%%%
%%%%%%%%%%%%%%%%%%%%%%%%%%%%%%%%%%%%%%%%%%%%%%%%%%%%%%%%%%%%%%%%%%%%%%%%%%%%%%%

%%%%%%%%%%%%%%%%%%%%%%%%%%%%%%%%%%%%%%%%%%%%%%%%%%%%%%%%%%%%%%%%%%
% Fonts
% -------------------------------------------------------
% Best LaTeX fonts: https://r2src.github.io/top10fonts/
% https://www.quora.com/Which-is-a-good-font-choice-for-writing-a-PhD-thesis-I-have-provisionally-chosen-CharterITC-but-I-am-not-sure-about-it-I-think-the-best-way-to-go-is-a-serif-font-but-I-dont-want-a-classic-look-but-something-more-updated
%https://tex.stackexchange.com/questions/59702/suggest-a-nice-font-family-for-my-basic-latex-template-text-and-math/59706
% https://www.overleaf.com/learn/latex/Font_typefaces
%%%%%%%%%%%%%%%%%%%%%%%%%%%%%%%%%%%%%%%%%%%%%%%%%%%%%%%%%%%%

%\usepackage{libertine} % nice font that works with the \oe. Breaks the cover page though.
%\usepackage{mathpazo} % using the Palatino font (nice font but does not work with the \oe and breaks cover page)
%\usepackage{palatino} % nice font but does not work with the \oe... and no math support
%\usepackage{fouriernc} % font with math support: New Century Schoolbook (nice font but does not work with the \oe...)
%\usepackage{charter} % using the Charter font (nice font but does not work with the \oe...)

\usepackage{fourier} % using the Utopia font (nice font but does not work with the \oe...)

%%%%%%%%%%%%%%%%%%%%%%%%
% Formatting
%%%%%%%%%%%%%%%%%%%%

\usepackage{calc} % simple arithmetics in latex commands
\usepackage{soul} % hyphenation for letterspacing, underlining, etc.

\makeatletter
\newlength\dlf@normtxtw
\setlength\dlf@normtxtw{\textwidth}
\newsavebox{\feline@chapter}
\newcommand\feline@chapter@marker[1][4cm]{%
	\sbox\feline@chapter{%
		\resizebox{!}{#1}{\fboxsep=1pt%
			\colorbox{gray}{\color{white}\thechapter}%
		}}%
		\rotatebox{90}{%
			\resizebox{%
				\heightof{\usebox{\feline@chapter}}+\depthof{\usebox{\feline@chapter}}}%
			{!}{\scshape\so\@chapapp}}\quad%
		\raisebox{\depthof{\usebox{\feline@chapter}}}{\usebox{\feline@chapter}}%
}

\newcommand\feline@chm[1][4cm]{%
	\sbox\feline@chapter{\feline@chapter@marker[#1]}%
	\makebox[0pt][c]{% aka \rlap
		\makebox[1cm][r]{\usebox\feline@chapter}%
	}}

\makechapterstyle{daleifmodif}{
\renewcommand\chapnamefont{\normalfont\Large\scshape\raggedleft\so}
\renewcommand\chaptitlefont{\normalfont\Large\bfseries\scshape}
\renewcommand\chapternamenum{} \renewcommand\printchaptername{}
\renewcommand\printchapternum{\null\hfill\feline@chm[2.5cm]\par}
\renewcommand\afterchapternum{\par\vskip\midchapskip}
\renewcommand\printchaptertitle[1]{\color{gray}\chaptitlefont\raggedleft
  \renewcommand\chaptername{Chapter}
  ##1\par}
}

\makeatother
\chapterstyle{daleifmodif}

% The pages should be numbered consecutively at the bottom centre of the page
\makepagestyle{myvf}
\makeoddfoot{myvf}{}{\thepage}{}
\makeevenfoot{myvf}{}{\thepage}{}
\makeheadrule{myvf}{\textwidth}{\normalrulethickness}
\makeevenhead{myvf}{\small\textsc{\leftmark}}{}{}
\makeoddhead{myvf}{}{}{\small\textsc{\rightmark}}
\pagestyle{myvf}

%%%%%%%%%%%%%%%%%%%%%%%%%%%%%%%%%%%%%%%%%%%%%%%%%%%%%%%%
% Create summary box to put at the end of each chapter %
%%%%%%%%%%%%%%%%%%%%%%%%%%%%%%%%%%%%%%%%%%%%%%%%%%%%%%%%

\newtcolorbox[]{summary}[2][]{fonttitle=\bfseries,title=#2,#1}

%%%%%%%%%%%%%%%%%%%%%%%%%%%%%%%%%%%%%%
% Create explanation Box environment %
%%%%%%%%%%%%%%%%%%%%%%%%%%%%%%%%%%%%%%

\tcbuselibrary{skins,breakable}

\newtcolorbox[auto counter, number within = chapter, number freestyle = {\noexpand\thechapter.\noexpand\arabic{\tcbcounter}}]{mybox}[2][]{%
    enhanced,
    fonttitle = \bfseries,
    title = Box~\thetcbcounter: #2,
    #1
    }

%%%%%%%%%%%%%%%%%%%%%%%%%%%%%%%%%%%%%
% Get the number of current chapter %
%%%%%%%%%%%%%%%%%%%%%%%%%%%%%%%%%%%%%

\newcommand\getcurrentref[1]{
 \ifnumequal{\value{#1}}{0}
  {??}
  {\the\value{#1}}
}

%%%%%%%%%%%%%%%%%%%%%%%%
% Insert an empty page %
%%%%%%%%%%%%%%%%%%%%%%%%

\usepackage{afterpage} % executes command after the next page break

\newcommand\blankpage{%
    \null
    \thispagestyle{empty}%
    %\addtocounter{page}{-1}% % uncomment to increase page counter
    \newpage
    }

\newcommand{\clearemptydoublepage}{\newpage{\thispagestyle{empty}\cleardoublepage}}

%%%%%%%%%%%%%%%%%%
% Epigraph style %
%%%%%%%%%%%%%%%%%%

\usepackage{epigraph} % provides commands to assist in the typesetting of a single epigraph

\setlength\epigraphwidth{1\textwidth}
\setlength\epigraphrule{0pt} % no line between
\setlength\beforeepigraphskip{1\baselineskip} % space before and after epigraph
\setlength\afterepigraphskip{2\baselineskip}
\renewcommand*{\textflush}{flushright}
\renewcommand*{\epigraphsize}{\normalsize\itshape}

%%%%%%%%%%%%%%%%%%%%%
% Break url in LOFs %
%%%%%%%%%%%%%%%%%%%%%

\PassOptionsToPackage{hyphens}{url}
\usepackage{hyperref} % extensive support for hypertext in latex

%%%%%%%%%%%%%%%%%%%%%%%%%%%
% Fixing landscape tables %
%%%%%%%%%%%%%%%%%%%%%%%%%%%

% solution from https://github.com/crsh/papaja/issues/287

% Manuscript styling
\usepackage{csquotes} % provides advanced facilities for inline and display quotations.
\usepackage{upgreek} % provides the upright Greek letters from the Euler or Adobe Symbol fonts
\captionsetup{font=singlespacing,justification=justified}

% Table formatting
% \usepackage{longtable} % allows writing tables that extend on several pages
% \usepackage{lscape} % figures or tables in landscape mode
% \usepackage[counterclockwise]{rotating} % Landscape page setup for large tables
\usepackage{multirow}	% table styling
\usepackage{tabularx}	% control column width
\usepackage[flushleft]{threeparttable} % allows for three part tables with a specified notes section
\usepackage{threeparttablex} % lets threeparttable work with longtable

% Create new environments so endfloat can handle them
% \newenvironment{ltable}
%   {\begin{landscape}\begin{center}\begin{threeparttable}}
%   {\end{threeparttable}\end{center}\end{landscape}}
\newenvironment{lltable}{\begin{landscape}\begin{center}\begin{ThreePartTable}}{\end{ThreePartTable}\end{center}\end{landscape}}

% Enables adjusting longtable caption width to table width
% Solution found at http://golatex.de/longtable-mit-caption-so-breit-wie-die-tabelle-t15767.html
\makeatletter
\newcommand\LastLTentrywidth{1em}
\newlength\longtablewidth
\setlength{\longtablewidth}{1in}
\newcommand{\getlongtablewidth}{\begingroup \ifcsname LT@\roman{LT@tables}\endcsname \global\longtablewidth=0pt \renewcommand{\LT@entry}[2]{\global\advance\longtablewidth by ##2\relax\gdef\LastLTentrywidth{##2}}\@nameuse{LT@\roman{LT@tables}} \fi \endgroup}

%%%%%%%%%%%%%%%%%%%%%%%%%%%
% My (APA6) caption style %
%%%%%%%%%%%%%%%%%%%%%%%%%%%

\usepackage{caption} % provides many ways to customise captions in floating environments
\captionsetup[figure]{labelsep=period,labelfont={it,normalsize},textfont={normalsize}}

%%%%%%%%%%%%%%%%%%%%%%%%%%%%%%%%%%%%%%%%%%%%%%%%%%%%%%%%%%%
% Removing extra vertical blank spaces between paragraphs %
%%%%%%%%%%%%%%%%%%%%%%%%%%%%%%%%%%%%%%%%%%%%%%%%%%%%%%%%%%%

\raggedbottom
