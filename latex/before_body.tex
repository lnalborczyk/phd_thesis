%%%%%%%%%%%%%%%%%%%%%
% Cover information %
%%%%%%%%%%%%%%%%%%%%%

\Specialite{Cognitive Sciences, Psychology, Neurocognition \& \\ Educational Sciences}
\Arrete{May 25, 2016}
\Auteur{Ladislas \textsc{Nalborczyk}}
\Directeur{Dr. Hélène \textsc{L\oe venbruck} \& Pr. Ernst \textsc{Koster}}
\CoDirecteur{Dr. Marcela \textsc{Perrone-Bertolotti}}

% Lab(s)
\Laboratoire{Laboratoire de Psychologie et NeuroCognition (UGA) \& \\ Psychopathology and Affective Neuroscience Laboratory (UGent)}

% Doctoral school(s)
\EcoleDoctorale{the Doctoral School EDISCE (UGA) and the Doctoral School of Social and Behavioural Sciences (UGent)}

% Title
\Titre{Understanding rumination as\\a form of inner speech}

% Sub-title
% \Soustitre{Basic and clinical aspects}

% Defense date
\Depot{October 20, 2019}

\Jury{

%\UGTPresident{XXXX}{...}
%\UGTPresidente{XXXX}{...}

\UGTRapporteur{XXXX}{...}
\UGTRapporteur{XXXX}{...}

\UGTExaminatrice{XXXX}{...}
\UGTExaminatrice{XXXX}{...}

\UGTDirecteur{Dr. Hélène \textsc{L\oe venbruck}}{\textsc{CR, CNRS, Université Grenoble Alpes}}
\UGTDirecteur{Pr. Ernst \textsc{Koster}}{\textsc{Professor, Ghent University}}
\UGTCoDirecteur{Dr. Marcela \textsc{Perrone-Bertolotti}}{\textsc{MCF, Université Grenoble Alpes}}
}

%%%%%%%%%%%%%%%%%%%%%%%%%%%%%%%%%%%%%%%
%%%%% end of title page template %%%%%%
%%%%%%%%%%%%%%%%%%%%%%%%%%%%%%%%%%%%%%%

\pagenumbering{roman} % start page numbering (roman style)
\MakeUGthesePDG % make cover page
\blankpage % leave an empty page between cover page and TOC
