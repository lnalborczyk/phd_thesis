\documentclass[a4paper,12pt,oneside,oldfontcommands]{memoir}
\usepackage{lmodern}
\usepackage{amssymb,amsmath}
\usepackage{ifxetex,ifluatex}
\usepackage{fixltx2e} % provides \textsubscript
\ifnum 0\ifxetex 1\fi\ifluatex 1\fi=0 % if pdftex
  \usepackage[T1]{fontenc}
  \usepackage[utf8]{inputenc}
\else % if luatex or xelatex
  \ifxetex
    \usepackage{mathspec}
  \else
    \usepackage{fontspec}
  \fi
  \defaultfontfeatures{Ligatures=TeX,Scale=MatchLowercase}
\fi
% use upquote if available, for straight quotes in verbatim environments
\IfFileExists{upquote.sty}{\usepackage{upquote}}{}
% use microtype if available
\IfFileExists{microtype.sty}{%
\usepackage{microtype}
\UseMicrotypeSet[protrusion]{basicmath} % disable protrusion for tt fonts
}{}
\usepackage[margin=1in]{geometry}
\usepackage{hyperref}
\hypersetup{unicode=true,
            pdftitle={Psychophysiological characteristics of verbal rumination},
            pdfborder={0 0 0},
            breaklinks=true}
\urlstyle{same}  % don't use monospace font for urls
\usepackage{natbib}
\bibliographystyle{apalike}
\usepackage{color}
\usepackage{fancyvrb}
\newcommand{\VerbBar}{|}
\newcommand{\VERB}{\Verb[commandchars=\\\{\}]}
\DefineVerbatimEnvironment{Highlighting}{Verbatim}{commandchars=\\\{\}}
% Add ',fontsize=\small' for more characters per line
\usepackage{framed}
\definecolor{shadecolor}{RGB}{248,248,248}
\newenvironment{Shaded}{\begin{snugshade}}{\end{snugshade}}
\newcommand{\KeywordTok}[1]{\textcolor[rgb]{0.13,0.29,0.53}{\textbf{#1}}}
\newcommand{\DataTypeTok}[1]{\textcolor[rgb]{0.13,0.29,0.53}{#1}}
\newcommand{\DecValTok}[1]{\textcolor[rgb]{0.00,0.00,0.81}{#1}}
\newcommand{\BaseNTok}[1]{\textcolor[rgb]{0.00,0.00,0.81}{#1}}
\newcommand{\FloatTok}[1]{\textcolor[rgb]{0.00,0.00,0.81}{#1}}
\newcommand{\ConstantTok}[1]{\textcolor[rgb]{0.00,0.00,0.00}{#1}}
\newcommand{\CharTok}[1]{\textcolor[rgb]{0.31,0.60,0.02}{#1}}
\newcommand{\SpecialCharTok}[1]{\textcolor[rgb]{0.00,0.00,0.00}{#1}}
\newcommand{\StringTok}[1]{\textcolor[rgb]{0.31,0.60,0.02}{#1}}
\newcommand{\VerbatimStringTok}[1]{\textcolor[rgb]{0.31,0.60,0.02}{#1}}
\newcommand{\SpecialStringTok}[1]{\textcolor[rgb]{0.31,0.60,0.02}{#1}}
\newcommand{\ImportTok}[1]{#1}
\newcommand{\CommentTok}[1]{\textcolor[rgb]{0.56,0.35,0.01}{\textit{#1}}}
\newcommand{\DocumentationTok}[1]{\textcolor[rgb]{0.56,0.35,0.01}{\textbf{\textit{#1}}}}
\newcommand{\AnnotationTok}[1]{\textcolor[rgb]{0.56,0.35,0.01}{\textbf{\textit{#1}}}}
\newcommand{\CommentVarTok}[1]{\textcolor[rgb]{0.56,0.35,0.01}{\textbf{\textit{#1}}}}
\newcommand{\OtherTok}[1]{\textcolor[rgb]{0.56,0.35,0.01}{#1}}
\newcommand{\FunctionTok}[1]{\textcolor[rgb]{0.00,0.00,0.00}{#1}}
\newcommand{\VariableTok}[1]{\textcolor[rgb]{0.00,0.00,0.00}{#1}}
\newcommand{\ControlFlowTok}[1]{\textcolor[rgb]{0.13,0.29,0.53}{\textbf{#1}}}
\newcommand{\OperatorTok}[1]{\textcolor[rgb]{0.81,0.36,0.00}{\textbf{#1}}}
\newcommand{\BuiltInTok}[1]{#1}
\newcommand{\ExtensionTok}[1]{#1}
\newcommand{\PreprocessorTok}[1]{\textcolor[rgb]{0.56,0.35,0.01}{\textit{#1}}}
\newcommand{\AttributeTok}[1]{\textcolor[rgb]{0.77,0.63,0.00}{#1}}
\newcommand{\RegionMarkerTok}[1]{#1}
\newcommand{\InformationTok}[1]{\textcolor[rgb]{0.56,0.35,0.01}{\textbf{\textit{#1}}}}
\newcommand{\WarningTok}[1]{\textcolor[rgb]{0.56,0.35,0.01}{\textbf{\textit{#1}}}}
\newcommand{\AlertTok}[1]{\textcolor[rgb]{0.94,0.16,0.16}{#1}}
\newcommand{\ErrorTok}[1]{\textcolor[rgb]{0.64,0.00,0.00}{\textbf{#1}}}
\newcommand{\NormalTok}[1]{#1}
\usepackage{longtable,booktabs}
\usepackage{graphicx,grffile}
\makeatletter
\def\maxwidth{\ifdim\Gin@nat@width>\linewidth\linewidth\else\Gin@nat@width\fi}
\def\maxheight{\ifdim\Gin@nat@height>\textheight\textheight\else\Gin@nat@height\fi}
\makeatother
% Scale images if necessary, so that they will not overflow the page
% margins by default, and it is still possible to overwrite the defaults
% using explicit options in \includegraphics[width, height, ...]{}
\setkeys{Gin}{width=\maxwidth,height=\maxheight,keepaspectratio}
\IfFileExists{parskip.sty}{%
\usepackage{parskip}
}{% else
\setlength{\parindent}{0pt}
\setlength{\parskip}{6pt plus 2pt minus 1pt}
}
\setlength{\emergencystretch}{3em}  % prevent overfull lines
\providecommand{\tightlist}{%
  \setlength{\itemsep}{0pt}\setlength{\parskip}{0pt}}
\setcounter{secnumdepth}{5}
% Redefines (sub)paragraphs to behave more like sections
\ifx\paragraph\undefined\else
\let\oldparagraph\paragraph
\renewcommand{\paragraph}[1]{\oldparagraph{#1}\mbox{}}
\fi
\ifx\subparagraph\undefined\else
\let\oldsubparagraph\subparagraph
\renewcommand{\subparagraph}[1]{\oldsubparagraph{#1}\mbox{}}
\fi

%%% Use protect on footnotes to avoid problems with footnotes in titles
\let\rmarkdownfootnote\footnote%
\def\footnote{\protect\rmarkdownfootnote}

%%% Change title format to be more compact
\usepackage{titling}

% Create subtitle command for use in maketitle
\providecommand{\subtitle}[1]{
  \posttitle{
    \begin{center}\large#1\end{center}
    }
}

\setlength{\droptitle}{-2em}

  \title{Psychophysiological characteristics of verbal rumination}
    \pretitle{\vspace{\droptitle}\centering\huge}
  \posttitle{\par}
    \author{}
    \preauthor{}\postauthor{}
    \date{}
    \predate{}\postdate{}
  
%%%%%%%%%%%%%%%%%%%%%%%%%%%%%
% Loading relevant packages %
%%%%%%%%%%%%%%%%%%%%%%%%%%%%%

\usepackage{tikz}
\usepackage{lscape}
\usepackage{indentfirst}
\usepackage{float}
\usepackage[flushleft]{threeparttable}
\usepackage[fulladjust]{marginnote}
\usepackage{tcolorbox}
\usepackage{pdfpages}
\usetikzlibrary{intersections}
\tcbuselibrary{listings,breakable}
\usepackage{pifont}
\usepackage{hyperref}
\usepackage[utf8]{inputenc}
\usepackage{graphicx,pdflscape}
\usepackage{geometry}
\usepackage{float}
\usepackage{longtable}
\usepackage{supertabular}
\usepackage{subfig}
\usepackage{scrextend}
\usepackage{tabularx}
\usepackage{lscape}
\usepackage{tabu}
\usepackage{array}
\usepackage[gen]{eurosym}
\usepackage{subfig}
\usepackage{stackrel,amssymb}
\usepackage{textcomp}
\usepackage{setspace}

\usepackage{microtype} % make better looking pdf

\usepackage{booktabs,caption,fixltx2e}
\usepackage[none]{hyphenat}

\usepackage[a4paper]{meta-donnees} % cover page
\usepackage{lmodern}

%%%%%%%%%%%%%%%%%%%%%%%%%%%%%%%%%%%%%%%%%%%%%%%%%%%%%%%%%%%%%%%%%%%%%
% below is the by-default configuration of bookdown-demo            %
% https://github.com/rstudio/bookdown-demo/blob/master/preamble.tex %
%%%%%%%%%%%%%%%%%%%%%%%%%%%%%%%%%%%%%%%%%%%%%%%%%%%%%%%%%%%%%%%%%%%%%

\usepackage{graphicx}
\usepackage{amsthm}

\makeatletter

\def\thm@space@setup{
  \thm@preskip=8pt plus 2pt minus 4pt
  \thm@postskip=\thm@preskip
}

\makeatother

%%%%%%%%%%%%%%%%%%%%%%%%
% remove default title %
% https://stackoverflow.com/questions/45963505/coverpage-and-copyright-notice-before-title-in-r-bookdown
%%%%%%%%%%%%%%%%%%%%%%%%

\let\oldmaketitle\maketitle
\AtBeginDocument{\let\maketitle\relax}

%%%%%%%%%%%%%%%%%%%%%%%%%%%%%%%%%%%%%%%%%%%%%%%%%%%%%%%%%%%%%
% Add a lettrine to the very first character of the content %
%%%%%%%%%%%%%%%%%%%%%%%%%%%%%%%%%%%%%%%%%%%%%%%%%%%%%%%%%%%%M

\usepackage{lettrine}
\newcommand{\initial}[1]{
	\lettrine[lines=3,lhang=0.33,nindent=0em]{
		\color{gray}
     		{\textsc{#1}}}{}}

%%%%%%%%%%%%%%%%%%%
% Font and format %
%%%%%%%%%%%%%%%%%%%

% Font with math support: New Century Schoolbook
\usepackage{fouriernc}
\usepackage[T1]{fontenc}

%%%%%%%%%%%%%%%%%%%%%%%%%%%%%%%%%%%%%%%%%%%%%%%%%%%%%%%%%%%%%
% https://texdoc.net/texmf-dist/doc/latex/memoir/memman.pdf %
%%%%%%%%%%%%%%%%%%%%%%%%%%%%%%%%%%%%%%%%%%%%%%%%%%%%%%%%%%%%%

%\setstocksize{11in}{8.5in} % "real" paper size, already set by a4paper class option
%\settrimmedsize{11in}{8.5in}{*} % trimmed paper size (trimmed on the left and right)

% Spine and trim page margins from main typeblock
%\setlrmarginsandblock{25mm}{25mm}{*}

% Top and bottom page margins from main typeblock
%\setulmarginsandblock{15mm}{20mm}{*}

%\setlrmargins{25mm}{25mm}{*}
%\setulmargins{1.0in}{*}{*}

%\setheadfoot{13pt}{26pt}
%\setheaderspaces{*}{13pt}{*} % set space between header and block

% apply and enforce the layout
%\checkandfixthelayout

%%%%%%%%%%%%%%%%%%%%%%%%%%%%%%%%%%%%%%%%%%%%%%%%
% alternative format
%%%%%%%%%%%%%%%%%%%%%%%%%%%%%%%%%%%%%%%%%%%%%%%%
% \settrimmedsize{297mm}{210mm}{*}
% \setlength{\trimtop}{0pt}
% \setlength{\trimedge}{\stockwidth}
% \addtolength{\trimedge}{-\paperwidth}
% \settypeblocksize{634pt}{448.13pt}{*}
% \setulmargins{4cm}{*}{*}
% \setlrmargins{*}{*}{1.5}
% \setmarginnotes{17pt}{51pt}{\onelineskip}
% \setheadfoot{\onelineskip}{2\onelineskip}
% \setheaderspaces{*}{2\onelineskip}{*}
% \checkandfixthelayout
%%%%%%%%%%%%%%%%%%%%%%%%%%%%%%%%%%%%%%%%%%%%%%%%

% ensure single spacing
\frenchspacing

%%%%%%%%%%%%%%%%%%%%%%%%%%%%%%%%%%%%%%%%%%%%%%%%%%%%%%%%%%%%%%%%%%%%%%%%%%%%%%%
% Chapter style (taken and slightly modified from Lars Madsen Memoir Chapter) %
%%%%%%%%%%%%%%%%%%%%%%%%%%%%%%%%%%%%%%%%%%%%%%%%%%%%%%%%%%%%%%%%%%%%%%%%%%%%%%%

\usepackage{calc,soul,fourier}
\makeatletter
\newlength\dlf@normtxtw
\setlength\dlf@normtxtw{\textwidth}
\newsavebox{\feline@chapter}
\newcommand\feline@chapter@marker[1][4cm]{%
	\sbox\feline@chapter{%
		\resizebox{!}{#1}{\fboxsep=1pt%
			\colorbox{gray}{\color{white}\thechapter}%
		}}%
		\rotatebox{90}{%
			\resizebox{%
				\heightof{\usebox{\feline@chapter}}+\depthof{\usebox{\feline@chapter}}}%
			{!}{\scshape\so\@chapapp}}\quad%
		\raisebox{\depthof{\usebox{\feline@chapter}}}{\usebox{\feline@chapter}}%
}

\newcommand\feline@chm[1][4cm]{%
	\sbox\feline@chapter{\feline@chapter@marker[#1]}%
	\makebox[0pt][c]{% aka \rlap
		\makebox[1cm][r]{\usebox\feline@chapter}%
	}}

\makechapterstyle{daleifmodif}{
\renewcommand\chapnamefont{\normalfont\Large\scshape\raggedleft\so}
\renewcommand\chaptitlefont{\normalfont\Large\bfseries\scshape}
\renewcommand\chapternamenum{} \renewcommand\printchaptername{}
\renewcommand\printchapternum{\null\hfill\feline@chm[2.5cm]\par}
\renewcommand\afterchapternum{\par\vskip\midchapskip}
\renewcommand\printchaptertitle[1]{\color{gray}\chaptitlefont\raggedleft
  \renewcommand\chaptername{Chapter}
  ##1\par}
}

\makeatother
\chapterstyle{daleifmodif}

% The pages should be numbered consecutively at the bottom centre of the page
\makepagestyle{myvf}
\makeoddfoot{myvf}{}{\thepage}{}
\makeevenfoot{myvf}{}{\thepage}{}
\makeheadrule{myvf}{\textwidth}{\normalrulethickness}
\makeevenhead{myvf}{\small\textsc{\leftmark}}{}{}
\makeoddhead{myvf}{}{}{\small\textsc{\rightmark}}
\pagestyle{myvf}

%%%%%%%%%%%%%%%%%%%%%%%%%%%%%%%%%%%%%%%%%%%%%%%%%%%%%%%%
% Create summary box to put at the end of each chapter %
%%%%%%%%%%%%%%%%%%%%%%%%%%%%%%%%%%%%%%%%%%%%%%%%%%%%%%%%

\newtcolorbox[]{summary}[2][]{title=#2,#1}

%%%%%%%%%%%%%%%%%%%%%%%%%%%%%%%%%%%%%
% Get the number of current chapter %
%%%%%%%%%%%%%%%%%%%%%%%%%%%%%%%%%%%%%

\newcommand\getcurrentref[1]{
 \ifnumequal{\value{#1}}{0}
  {??}
  {\the\value{#1}}
}

%%%%%%%%%%%%%%%%%%%%%%%%
% Insert an empty page %
%%%%%%%%%%%%%%%%%%%%%%%%

\usepackage{afterpage}

\newcommand\blankpage{%
    \null
    \thispagestyle{empty}%
    %\addtocounter{page}{-1}% % uncomment to increase page counter
    \newpage}

\begin{document}
\maketitle

%%%%%%%%%%%%%%%%%%%%%
% Cover information %
%%%%%%%%%%%%%%%%%%%%%

\Specialite{Cognitive Sciences, Psychology, Neurocognition \& \\ Educational Sciences}
\Arrete{May 25, 2016}
\Auteur{Ladislas \textsc{Nalborczyk}}
\Directeur{Dr. Hélène \textsc{L\oe venbruck} \& Pr. Ernst \textsc{Koster}}
\CoDirecteur{Dr. Marcela \textsc{Perrone-Bertolotti}}

% Lab(s)
\Laboratoire{Laboratoire de Psychologie et NeuroCognition (UGA) \& Psychopathology and Affective Neuroscience Laboratory (UGent)}

% Doctoral school(s)
\EcoleDoctorale{the doctoral school EDISCE (UGA) and the Doctoral School of Social and Behavioural Sciences (UGent)}

% Title
\Titre{Psychophysiological characteristics\\ of verbal rumination}

% Sub-title
% \Soustitre{Basic and clinical aspects}

% Defense date
\Depot{October 20, 2019}

\Jury{

%\UGTPresident{XXXX}{...}
%\UGTPresidente{XXXX}{...}

\UGTRapporteur{XXXX}{...}
\UGTRapporteur{XXXX}{...}

\UGTExaminatrice{XXXX}{...}
\UGTExaminatrice{XXXX}{...}

\UGTDirecteur{Dr. Hélène \textsc{L\oe venbruck}}{\textsc{CR, CNRS, Université Grenoble Alpes}}
\UGTDirecteur{Pr. Ernst \textsc{Koster}}{\textsc{Professor, Ghent University}}
\UGTCoDirecteur{Dr. Marcela \textsc{Perrone-Bertolotti}}{\textsc{MCF, Université Grenoble Alpes}}
}

%%%%%%%%%%%%%%%%%%%%%%%%%%%%%%%%%%%%%%%
%%%%% end of title page template %%%%%%
%%%%%%%%%%%%%%%%%%%%%%%%%%%%%%%%%%%%%%%

\pagenumbering{roman} % start page numbering (roman style)
\MakeUGthesePDG % make cover page
\blankpage % leave an empty page between cover page and TOC

{
\setcounter{tocdepth}{2}
\tableofcontents
}
\listoftables
\listoffigures
\chapter*{Résumé}\label{resume}
\addcontentsline{toc}{chapter}{Résumé}

\initial{B}lah blah\ldots{}

\afterpage{\blankpage}

\chapter*{Overzicht}\label{overzicht}
\addcontentsline{toc}{chapter}{Overzicht}

\initial{B}lah blah\ldots{}

\chapter*{Abstract}\label{abstract}
\addcontentsline{toc}{chapter}{Abstract}

\initial{B}lah blah\ldots{}

\afterpage{\blankpage}

\chapter*{Acknowledgements}\label{acknowledgements}
\addcontentsline{toc}{chapter}{Acknowledgements}

\initial{A}cknowledgements will be included in the final version of this
thesis.

\afterpage{\blankpage}

\chapter*{Preface}\label{preface}
\addcontentsline{toc}{chapter}{Preface}

\initial{B}lah blah\ldots{}

\afterpage{\blankpage}

\part{Theoretical chapters}\label{part-theoretical-chapters}

\chapter{Theoretical framework}\label{intro}

\pagenumbering{arabic}

\initial{A}s you read these words, you might notice the presence of a
familiar companion. A voice-like phenomenon that remains unnoticed until
we pay attention to it. However, if I ask you to focus on that little
voice while you are reading these lines, you would probably be able to
provide a relatively fine-graind description of this thing that we call
inner speech. Whose voice is it ? Is it yours ? Is it gendered ? It is
usually possible to examine these aspects as well as more low-level
features like the tone of this soundy companion, its pitch, its tempo,
or virtually any sensory aspect of it. This first set of very basic
observations already provide us some very important insights. First, if
we can think about inner speech, then it should be something different
from thinking itself (réf ?). Rather, inner speech (or \emph{covert
speech}), can be construed as \emph{a} vehicle for conscious thought
(instead of \emph{verbal thinking}, for instance)\footnote{We will not
  dwell on the touchy question whether inner speech is a necessary
  condition for consciousness. For the current purpose, it is sufficient
  to say that thinking and inner speech are ontologically separable.}.
Second, the set of observations we can make about our inner voice also
tautologically reveals that inner speech is accompanied by sensory
percepts (sounds, kinaesthesic feelings, etc.). It thus raises an
interesting question: where do these percepts come from ? Why do they
look like the one we experience when we \emph{actually} (overtly) speak
?

This first set of questions refer to the \emph{nature} of inner speech,
that is, \emph{what} is it ? In the current work, we are mostly
concerned with this first question. Another related set of interesting
questions revolve around the \emph{function} question, that is,
\emph{what for} is it ? The influential Vygotskian theory of inner
speech development suggests that inner speech evolves from \emph{private
speech} (i.e., self-adressed overt speech) during childhood. As such, we
(as others have argued elsewhere) postulate that the functions of inner
speech are inherited from the functions of private speech, via a
mechanism of internalisation. The specific features of this
internalisation processs are worthy of investigation on their own (and
we briefly discuss them later on) but we are mostly interesred in the
\emph{what is} (the nature) question here. Thus, we will only sparsely
address the \emph{functions} question in the following text.

That being said, it is interesting to look at situations in which these
functions do not work as intended. These \emph{dysfunctions} \citep[that
can also be considered as
\emph{mis-exadaptation},][]{agnati_possible_2012} are as spread
as\ldots{} They can generally be understood as transdiagnostic processes
(i.e., a process that is not specific to a single pathology), and cover
various\ldots{}

\ldots{}

Learning how to internalise speech might be similar to learning how to
internalise playing an instrument\ldots{} Let's consider the analogy
between speaking and playing an instrument (e.g., the piano). Playing
piano results from the learning of an infinitely complex coordination of
fine motor sequences, that in turn produce sensory (kinestheatic,
auditory, visual, etc) feedback to the producer of the action (the
agent). It seems that (from a certain level of analysis), the act of
speech can be paralleled with the act of playing an instrument in that
its consists in the coordination of infinitely complex movements that
result in some modifications in the environment that in turn generate
sensory feedbacks for the agent. Thus, pursuing the analogy, we argue
that imagining playing pian and imagining speaking (i.e., producing
inner speech) might rest on similar mechanisms\ldots{} see O'Shea \&
Moran (2018) on expert pianists\ldots{}

\section{Rumination as a form of repetitive negative
thinking}\label{rumination-as-a-form-of-repetitive-negative-thinking}

Blah blah \citep{koster_rumination_2013}\ldots{}

As suggested by \citet{Christoff2016}, rumination and other forms of
spontaneous thoughts can be considered in a common conceptual space (see
Figure 1). This space is built upon two dimensions: \emph{deliberate
constraints} and \emph{automatic constraints}. These dimensions
represent two general mechanisms that allow to constrain the contents of
these related mental states and the transitions between them. The first
contrain correspond to a deliberate processus and is implemented through
\textbf{cognitive control} \citep{Miller2000}. The second constrain is
referring to more automatic constrains like sensory afferences. In this
framework, rumination is characterizsd by the highest level of automatic
constraints and spread all along the \emph{deliberate constraints}
dimension.

\begin{figure}

{\centering \includegraphics[width=0.75\linewidth]{assets/conceptual_space} 

}

\caption{Conceptual space of different types of thought (Christoff et al., 2016)}\label{fig:conceptual}
\end{figure}

\textbf{Copy-pasted from zygoto old intro\ldots{}}

Speech production might be one of the most complex motor action ever
studied. Involving the fine-grained coordination of more than 100
muscles in the upper part of the body \citep{simonyan_laryngeal_2011},
its expression is shaped by a subtle combination of physiological,
cultural and evolutionary determinants. In adult humans, its covert
counterpart (i.e., \emph{inner speech} or \emph{verbal imagery}) has
developed to allow the full reconstruction of usual overt speech
situations. In the same way as visual imagery allows to mentally examine
visual scenes, \emph{verbal imagery} can be used as an internal tool,
allowing --amongst other things-- to rehearse or to prepare past and
future speech situations \citep[for a review,
see][]{perrone-bertolotti_what_2014}. In consideration of its
self-evident motoric nature, a parallel can be drawn between verbal
imagery and other forms of motor imagery (e.g., imagined walking or
imagined writing). As such, inner speech studies might benefit from
insights gained from the study of motor imagery and the field of motor
cognition \citep[e.g.,][]{haggard_conscious_2005, jeannerod_motor_2006}.
Whereas previous research have demonstrated that it is possible to
record muscle-specific electromyographic correlates of inner speech
production using invasive intramuscular needle electrodes, more recent
research using surface electromyography lead to mixed results. Building
upon previous work, we describe an experimental set-up using surface
electromyography with the aim of refining the description of the
involvement of the speech motor system during inner speech production.

\section{What is motor imagery ?}\label{what-is-motor-imagery}

\subsection{The motor simulation
theory}\label{the-motor-simulation-theory}

Motor imagery can be defined as the mental process by which one
rehearses a given action, without engaging in the physical movements
involved in this particular action. One of the most influential
theoretical explanation of this broad phenomenon, the \emph{motor
simulation theory}
\citep[MST;][]{jeannerod_representing_1994, jeannerod_neural_2001, jeannerod_motor_2006},
contains the three following postulates at its core: i) there exists a
continuum between the covert (the mental representation) and the overt
execution of an action, ii) action representations can operate off-line,
via a \emph{simulation} mechanism, and iii) covert actions rely on the
same set of mechanisms as the overt actions they simulate, except that
execution is inhibited \citep{oshea_does_2017}.

In this framework, the concept of simulation refers to the ``offline
rehearsal of neural networks'' \citep{jeannerod_motor_2006}, and motor
imagery is conceptualised as a simulation of the covert stage of the
same executed action \citep{oshea_does_2017}. The MST shares some
similarities with the theories of embodied and grounded cognition
\citep{barsalou_grounded_2008} in that both allow to account for the
phenomenon of motor imagery by appealing to a simulation mechanism.
However, the concept of simulation in grounded theories is assumed to be
multi-modal (not just motoric) and to operate in order to achieve a
particular abstract knowledge \citep{oshea_does_2017}, which is not the
concern of the MST\footnote{We should also make a distinction between
  \emph{embodiment of content}, which concerns the conceptual content of
  language, and \emph{embodiment of form}, which concerns ``the vehicle
  of thought'', that is, proper speech production
  \citep{pickering_integrated_2013}.}.

The MST is supported by a wealth a findings, going from mental
chronometry studies showing that the time taken to perform an action is
often found to be similar to the time needed to imagine the
corresponding action \citep[though not always, see][for a review of
controversial findings and for an alternative conceptualisation of motor
imagery]{glover_motor-cognitive_2017}, to neuroimaging and
neurostimulation studies showing that both motor imagery and overt
actions tend to recruit similar frontal, parietal and sub-cortical
regions \citep[e.g.,][]{hetu_neural_2013, jeannerod_neural_2001}. The
involvement of the motor system during motor imagery is also supported
by repeated observations of autonomic responses and peripheral muscular
activity during motor imagery (we discuss these observations in section
\ref{emg}).

\subsection{Emulation and internal
models}\label{emulation-and-internal-models}

A second class of explanatory models of motor imagery are concerned with
the phenomenon of \emph{emulation} and with \emph{internal models}
\citep[see][for a review of the similarities and dissimilarities of
simulation and emulation models]{gentsch_towards_2016}.

Internal model theories share the postulate that the motor system is
represented by \emph{internal models}, whose function is to estimate and
anticipate the outcome of a motor command. One of its variant, the
\emph{motor control theory}
\citep[e.g.,][]{kawato_internal_1999, wolpert_internal_1995}, assumes
two kind of models: a forward model that predicts the sensory
consequences of motor commands from efference copies, and an inverse
model that calculates the feed-forward motor commands from the desired
movement \citep{gentsch_towards_2016}.

Emulation theories
\citep[e.g.,][]{grush_emulation_2004, moulton_imagining_2009} borrow
from both previously discussed framework (i.e., simulation theories and
internal model theories) to posit a specific kind of simulation. While
the MST postulates that during simulation the motor system is guided
exclusively by internal motor representations, the emulation theories
suggest that both motor and sensory systems are emulated in parallel
\citep{grush_emulation_2004, oshea_does_2017}.

In the emulation model proposed by \citet{grush_emulation_2004}, the
\emph{emulator} is a device that implements the same input-output
function as the body (i.e., the musculoskeletal system and relevant
proprioceptive/kinaesthetic systems). When the emulator receives a copy
of the control signal (which is also sent to the body), it produces an
output signal (the emulator feedback), identical or similar to the
feedback signal produced by the body\footnote{In Grush's terminology,
  \emph{emulator} is used as a synonym for \emph{forward models}
  \citep[see][pages 378-379]{grush_emulation_2004}.}. This feedback
would be responsible for the presence of sensory percepts (e.g., visual,
auditory, kinaesthetic) during motor imagery.

One important difference between the emulation theory of motor imagery
and the MST though, is that the latter takes the mere activation of
efferent motor centres as being sufficient for explaining motor imagery,
while the emulation theory postulates that an emulator of the
musculoskeletal system is needed \citep[pages
384-385]{grush_emulation_2004}. \citet{grush_emulation_2004} suggested
an analogy to illustrate this difference: ``The emulation theory claims
that motor imagery is like a pilot sitting in a flight simulator, and
the pilot's efferent commands (hand and foot movements, etc.) are
translated into faux ``sensory'' information (instrument readings, mock
visual display) by the flight simulator which is essentially an emulator
of an aircraft. The simulation theory claims that just a pilot, moving
her hands and feet around but driving neither a real aircraft nor a
flight simulation, is sufficient for mock sensory information''.
Alternatively, in the words of \citet{moulton_imagining_2009},
instrumental simulations (à la Jeannerod) can be thought of as
\emph{first-order} simulations that imitate the content of the simulated
action, while emulative simulations can be thought of as
\emph{second-order} simulations that imitate both the content and the
processes that change the content.

\section{Electromyography of covert actions}\label{emg}

\subsection{Explanations for the presence of muscular activity during
motor
imagery}\label{explanations-for-the-presence-of-muscular-activity-during-motor-imagery}

Motor imagery has consistently been defined as the mental rehearsal of a
motor action without any overt movement. One consequence of this claim
is that, in order to prevent execution, the neural commands for muscular
contractions should be blocked at some level of the motor system by
active inhibitory mechanisms \citep[for a review,
see][]{guillot_imagining_2012}. Despite these inhibitory mechanisms,
there is now abundant evidence for peripheral muscular activation during
motor imagery \citep[for a review,
see][]{guillot_contribution_2005, guillot_imagining_2012}. As suggested
by \citet{jeannerod_representing_1994}, the incomplete inhibition of the
motor commands would provide a valid explanation to account for the
peripheral muscular activity observed during motor imagery. This idea
has been corroborated by studies of changes in the excitability of the
motor pathways during motor imagery tasks. \citet{bonnet_mental_1997}
measured spinal reflexes while participants were instructed to either
press a pedal with the foot or to simulate the same action mentally.
They observed that both H-reflexes and T-reflexes increased during motor
imagery, and that these increases correlated with the force of the
simulated pressure. Using transcranial magnetic stimulation and motor
evoked potentials (MEPs), several investigators observed muscle-specific
increases of MEPs during various motor imagery tasks, while no such
increase could be observed in antagonist muscles
\citep[e.g.,][]{fadiga_corticospinal_1999, rossini_corticospinal_1999}\footnote{As
  a side note, we should remark that these findings are consistent with
  both the simulation and the emulation views on motor imagery.}.

Interestingly, the dominant interpretation of the muscular correlates of
motor imagery at the beginning of the last century was that the
peripheral muscular activity observed during imagined actions was the
\emph{source} of the mental content. However, as explained by
\citet{jeannerod_motor_2006}, this interpretation of mental processes as
a consequence of peripheral feedback is now disproved, for instance by
the simple fact that many people can experiment motor imagery, without
any observable muscular activity\footnote{The \emph{peripheralist}
  interpretation has also been disproved by the heroic experiment
  carried out by \citet{smith_lack_1947}. Smith used d-tubocurarine
  (curare) to paralyse his own facial muscles in order to test this
  interpretation. He later reported that, while being paralysed, he was
  still able to think in words and to solve mathematical problems.}. In
the most recent theoretical explanations of motor imagery (e.g., MST,
emulation or internal models theories), the peripheral activity is
rather assumed to be a consequence of an incomplete inhibition of motor
output during the mental states involving motor simulation/emulation
(i.e., these views adhere to a \emph{centralist} interpretation of the
physiological correlates of inner speech).

\subsection{Controversial findings}\label{controversial-findings}

As reviewed in \citet{guillot_electromyographic_2010}, although there
are many observations showing a peripheral muscular activity during
motor imagery, there are also many studies failing to do so, or
reporting surprisingly high levels of inter-subject variability, with
some participants showing no muscular activity at all. Putting aside the
discussion on the exact nature and location of the inhibitory mechanisms
during motor imagery \citep[see][]{guillot_imagining_2012}, two main
explanations have been advanced to resolve these discrepancies. First,
the electromyographic activity recorded during motor imagery could be
moderated by the perspective taken in motor imagery. We usually make a
distinction between a first-person perspective or \emph{internal
imagery} (i.e., imagining an action as we would execute it) and a
third-person perspective or \emph{external imagery} (i.e., imagining an
action as an observer of this action), that seem to involve different
neural and cognitive processes. It has been shown that a first-person
perspective generally results in greater EMG activity than motor imagery
in a third-person perspective
\citep{hale_effects_1982, harris_effects_1986}. Second, some authors
postulated that the intensity of the EMG activity recorded during motor
imagery might be related to the individual ability to form an accurate
mental representation of the motor skill (i.e., the vividness of the
mental image). However, after reviewing the available evidence,
\citet{guillot_brain_2009} concluded that this is unlikely to be the
case. Alternatively, discrepancies in experimental design and
methodological choices (e.g., use of intramuscular versus surface
electromyography) could also explain these different results
\citep{guillot_electromyographic_2010}.

In the next section, we turn to a discussion of inner speech
conceptualised as a kind of motor (and sensory) imagery of speech, and
discuss the theoretical underpinnings of this proposition as well as the
available empirical evidence in its support.

\section{What is that little voice inside my head
?}\label{what-is-that-little-voice-inside-my-head}

\subsection{Inner speech as multimodal verbal
imagery}\label{inner-speech-as-multimodal-verbal-imagery}

While grasping the concept of a visual image appears as relatively
straightforward, it seems more difficult at first to grasp the concept
of a motor image, especially when it comes to verbal imagery. The
subjective experience of the tension that results from a given position
of the articulators and the covert production of an incompatible speech
sound permits to substantiate what a motor image is. For instance, it is
generally impossible to generate the image of the pronunciation of the
sound ``b'' while keeping the mouth wide opened
\citep[e.g.,][]{binet_psychologie_1886, stricker_studien_1880}. This
simple experiment allows defining imagined speech as the simulation of
the corresponding overt verbal content, where \emph{simulation} is meant
to be understood either as the off-line rehearsal of neural motor
networks involved in the overt action \citep{jeannerod_motor_2006}, or
in the terms of the emulation theories discussed previously\footnote{Translated
  to speech, the MST is similar to previous proposals such as the
  \emph{motor theory of voluntary thinking} \citep{cohen_motor_1986} or
  the hierarchical model of mental practice \citep{mackay_problem_1981}.}.

The model of wilful (voluntary) inner speech production introduced in
\citet{loevenbruck_cognitive_2018} goes one step further and, by
building on the models of speech motor control
\citep[e.g.,][]{houde_speech_2011, wolpert_internal_1995}, describes
inner speech as ``multi-modal acts with multi-sensory percepts stemming
from coarse multi-sensory goals''. In other words, the auditory and
kinaesthetic sensations perceived during inner speech prediction are
assumed to be the predicted sensory consequences of speech motor acts,
emulated by internal forward models, that use the efference copies
issued from an inverse model \citep[this proposal shares similarities
with the emulation model of motor imagery discussed
earlier,][]{grush_emulation_2004}.

\textbf{Fin de l'intro old de zygoto\ldots{}}

\section{Overt and imagined actions}\label{overt-and-imagined-actions}

Wittgenstein's (1953) famous query: ``When I raise my arm, what is left
after subtracting the fact that my arm raised?''. We posit that what is
left is an internal model (a representation) of what should happen if
and when my arm goes up (Jeannerod, 1999)\ldots{}

\subsection{Motor imagery}\label{motor-imagery}

Considerable experimental evidence has accumulated to suggest that
movement execution and MI share substantial overlap of active brain
regions (for review, see Guillot et al., 2012). Such apparent functional
equivalence supports the hypothesis that MI draws on the similar neural
networks that are used in actual perception and motor control
(Jeannerod, 1994; Grezes and Decety, 2001; Holmes and Collins,
2001)\ldots{}

See introduction of O'Shea (2017) phd thesis introduction\ldots{}

See Stinear's chapter in Guillot's book for intracortical and spinal
mechanisms involved during motor imagery (p.55-57).

\subsubsection{Simulation theories}\label{simulation-theories}

For Jeannerod (1995), motor imagery is necessarily first-perspective.
Third perspective imagery is imagery, but not MOTOR imagery\ldots{}
Motor representations are conceived here as `internal models' of the
goal of an action.

\subsubsection{Emulation theories}\label{emulation-theories}

\ldots{}

\subsubsection{Action representation and internal
models}\label{action-representation-and-internal-models}

Voir Jeannerod (2004), Wolpert el al. (1995), Wolpert \& Gharamani
(2000)\ldots{}

\subsection{Inner speech - what is this little voice in my head
?}\label{inner-speech---what-is-this-little-voice-in-my-head}

\ldots{}

The inner voice as the sensory consequence (prediction, see Loevenbruck
et al., 2018) of imagined speech. Analogy with raising the arm: what we
perceive when we imagine raising our arm are the sensory consequences
(e.g., visual) of what would happen if we actually raised our arm, these
are then kind of predictions. The same thing happens during inner speech
production: the inner voice is the predicted auditory consequence of
actual speech, except that it's predicted. The two actions might seem
very different, partly because of differences in the degree of
automaticity. Imagining raising our arm might need a
voluntary/deliberate/conscious (choose a word) intention (i.e., I want
to raise my arm \textgreater{} I raise my arm) while speech imagery
(i.e., inner speech) seems more automatic: we do not expression
consciously the intention to speak, we just speak\ldots{}

\subsubsection{MVTV Cohen (1986)}\label{mvtv-cohen-1986}

\ldots{}

\subsubsection{Predictive models}\label{predictive-models}

Learning how to internalise speech might be similar to learning how to
internalise playing an instrument\ldots{} Let's consider the analogy
between speaking and playing an instrument (e.g., the piano). Playing
piano results from the learning of an infinitely complex coordination of
fine motor sequences, that in turn produce sensory (kinestheatic,
auditory, visual, etc) feedback to the producer of the action (the
agent). It seems that (from a certain level of analysis), the act of
speech can be paralleled with the act of playing an instrument in that
its consists in the coordination of infinitely complex movements that
result in some modifications in the environment that in turn generate
sensory feedbacks for the agent. Thus, pursuing the analogy, we argue
that imagining playing pian and imagining speaking (i.e., producing
inner speech) might rest on similar mechanisms\ldots{} see O'Shea \&
Moran (2018) on expert pianists\ldots{}

\subsubsection{Loevenbruck et al.,
HMOSAIC}\label{loevenbruck-et-al.-hmosaic}

\ldots{}

\chapter{Methodological framework}\label{methodological-framework}

\section{Electromyographic correlates of speech
production}\label{electromyographic-correlates-of-speech-production}

\ldots{}

\subsection{Speech production
mechanisms}\label{speech-production-mechanisms}

\ldots{}

\subsection{Speech production muscles}\label{speech-production-muscles}

\ldots{}

\subsection{Muscular physiology}\label{muscular-physiology}

\ldots{}

\subsection{EMG signal}\label{emg-signal}

\section{EMG signal measures}\label{emg-signal-measures}

Muscular activity can be studied at different levels. At the cellular
level, using electrophysiological measures like micro-electrods
implanted in the cell, that allow direct measures of \textbf{action
potential}. At the segmental level, biomechanis study muscular activity
using surface sensors, positionned on the skin\ldots{}intermediate
levels\ldots{}

\subsection{Motor unit action
potential}\label{motor-unit-action-potential}

The \textbf{motor unit action potential} (MUAP) is the electric field
resulting from the sum of the electric fiels emitted by each fiber of
the motor unit. This train of action potentials will generate a
\emph{train} of MUAP, call \textbf{motor unit action potential trains}
(MUAPT). The electric potential generated by this field is highly
dependent of parameters such as the number of fibers, their length,
speed of conduction and position of the neuromuscular junction\ldots{}

\begin{figure}[H]

{\centering \includegraphics[width=1\linewidth]{assets/muap} 

}

\caption{Motor unit action potential representation.}\label{fig:unnamed-chunk-1}
\end{figure}

To sum up, the EMG signal results from a mixture of recruited motor
units\ldots{}

\subsection{Surface EMG}\label{surface-emg}

\ldots{}\emph{crosstalk} phenomenon \citep{de_luca_use_1997}. In reason
of the important\ldots{} of facial muscles, the EMG activity of one
recorded muscle generally does not represent the activity of a single
muscle but rather a mixture of\ldots{} \citet{Rapin2011}\ldots{}

\subsection{Basic signal processing}\label{basic-signal-processing}

\ldots{}the EMG signal is a stochastic signal\ldots{} In order to
illustrate what EMG signal looks like, we simulated EMG signal based on
a standard algorithm implemented in the \texttt{biosignalEMG} package
\citep{R-biosignalEMG}.

\begin{Shaded}
\begin{Highlighting}[]
\KeywordTok{library}\NormalTok{(biosignalEMG)}
\KeywordTok{library}\NormalTok{(tidyverse)}

\NormalTok{emg <-}\StringTok{ }\KeywordTok{syntheticemg}\NormalTok{(}
  \DataTypeTok{off.sd =} \DecValTok{1}\NormalTok{, }\DataTypeTok{on.sd =} \DecValTok{2}\NormalTok{, }\DataTypeTok{on.mode.pos =} \FloatTok{0.1}\NormalTok{, }\DataTypeTok{samplingrate =} \FloatTok{1e3}\NormalTok{, }\DataTypeTok{units =} \StringTok{"mV"}
\NormalTok{  )}\OperatorTok{$}\NormalTok{values}

\KeywordTok{ts.plot}\NormalTok{(}
\NormalTok{  emg, }\DataTypeTok{xlab =} \StringTok{"Time (samples)"}\NormalTok{, }\DataTypeTok{ylab =} \StringTok{"Raw EMG signal (mV)"}\NormalTok{,}
  \DataTypeTok{col =} \StringTok{"steelblue"}
\NormalTok{  )}
\end{Highlighting}
\end{Shaded}

\begin{figure}[H]

{\centering \includegraphics[width=1\linewidth]{02-chap2_files/figure-latex/unnamed-chunk-2-1} 

}

\caption{Simulated EMG signal.}\label{fig:unnamed-chunk-2}
\end{figure}

We usually rectify the EMG signal by taking its absolute value and
substracting the mean in order to correct for any offset (bias) present
in the raw data.

\begin{Shaded}
\begin{Highlighting}[]
\OperatorTok{>}\StringTok{ }\NormalTok{emg <-}\StringTok{ }\KeywordTok{abs}\NormalTok{(emg }\OperatorTok{-}\StringTok{ }\KeywordTok{mean}\NormalTok{(emg) )}
\OperatorTok{>}\StringTok{ }
\ErrorTok{>}\StringTok{ }\KeywordTok{ts.plot}\NormalTok{(}
\OperatorTok{+}\StringTok{   }\NormalTok{emg, }\DataTypeTok{xlab =} \StringTok{"Time (samples)"}\NormalTok{, }\DataTypeTok{ylab =} \StringTok{"Rectified EMG signal"}\NormalTok{,}
\OperatorTok{+}\StringTok{   }\DataTypeTok{col =} \StringTok{"steelblue"}
\OperatorTok{+}\StringTok{   }\NormalTok{)}
\end{Highlighting}
\end{Shaded}

\begin{figure}[H]

{\centering \includegraphics[width=1\linewidth]{02-chap2_files/figure-latex/unnamed-chunk-3-1} 

}

\caption{Rectified EMG signal.}\label{fig:unnamed-chunk-3}
\end{figure}

From there, two main measures can be used to represent the magnitude of
muscle activity\footnote{But see \citet{phinyomark_feature_2012} for
  other features that can be extracted from the surface EMG signals.}.
The first one is the \textbf{mean absolute value} (MAV):

\[MAV = \frac{1}{N} \sum_{n=1}^{N} | x_{n} |\]

which is computed over a specific interval and where \(|x_{n}|\) is the
absolute value of a datum of EMG in the data window. The unit of
measurement is \(mV\) or \(\mu V\), and the MAV calculation is generally
similar to the numerical formula for integration
\citep{kamen_essentials_2010}. The second one is the
\textbf{root-mean-square} (RMS) amplitude:

\[RMS = \sqrt \frac{1}{N} \sum_{n=1}^{N} | x^{2}_{n} |\]

where \(| x^{2}_{n} |\) is the squared value of each EMG datum and has
both physical and physiological meanings\ldots{}

\section{Statistical modelling
approach}\label{statistical-modelling-approach}

Describe the way we approach data analysis here\ldots{}

\section{Overview of the experimental
chapters}\label{overview-of-the-experimental-chapters}

\ldots{}

\part{Experimental
chapters}\label{part-experimental-chapters}

\chapter{Orofacial electromyographic correlates of induced verbal
rumination}\label{orofacial-electromyographic-correlates-of-induced-verbal-rumination}

Summary of the research\ldots{}\footnote{This experimental chapter is a
  published paper reformatted for the need of this thesis. Source:
  Nalborczyk, L., Perrone-Bertolotti, M., Baeyens, C., Grandchamp, R.,
  Polosan, M., Spinelli, E., \ldots{} L\oe venbruck, H. (2017).
  Orofacial Electromyographic Correlates of Induced Verbal Rumination.
  \emph{Biological Psychology, 127}, 53-63.
  \url{http://dx.doi.org/10.1016/j.biopsycho.2017.04.013}.}

\section{Introduction}\label{introduction}

As humans, we spend a considerable amount of time reflecting upon
ourselves, thinking about our own feelings, thoughts and behaviors.
Self-reflection enables us to create and clarify the meaning of past and
present experiences (Boyd \& Fales, 1983; Nolen-Hoeksema, Wisco, \&
Lyubomirsky, 2008). However, this process can lead to unconstructive
consequences when self-referent thoughts become repetitive, abstract,
evaluative, and self-critical \citep{Watkins2008}.

Indeed, rumination is most often defined as a repetitive and recursive
mode of responding to negative affect \citep{Rippere1977} or life
situations \citep{Robinson2003}. Although rumination is a common process
that can be observed in the general population \citep{Watkins2008}, it
has been most extensively studied in depression and anxiety. Depressive
rumination has been thoroughly studied by Susan Nolen-Hoeksema, who
developed the Response Style Theory
\citep[RST,][]{nolen-hoeksema_responses_1991}. According to the RST,
depressive rumination is characterized by an evaluative style of
processing that involves recurrent thinking about the causes, meanings,
and implications of depressive symptoms. Even though rumination can
involve several modalities (i.e., visual, sensory), it is a
predominantly verbal process
\citep{goldwin_concreteness_2012, mclaughlin_effects_2007}. In this
study, we focus on verbal rumination, which can be conceived of as a
particularly significant form of inner speech.

Inner speech or covert speech can be defined as silent verbal production
in one's mind or the activity of silently talking to oneself (Zivin,
1979). The nature of inner speech is still a matter of theoretical
debate \citep[see][ for a review]{Perrone-Bertolotti2014}. Two opposing
views have been proposed in the literature: the Abstraction view and the
Motor Simulation view. The Abstraction view describes inner speech as
unconcerned with articulatory or auditory simulations and as operating
on an amodal level. It has been described as ``condensed, abbreviated,
disconnected, fragmented, and incomprehensible to others'' (Vygotsky,
1987). It has been argued that important words or grammatical affixes
may be dropped in inner speech (Vygotsky, 1987) or even that the
phonological form or representation of inner words may be incomplete
(Sokolov, 1972; Dell \& Repka, 1992). MacKay (1992) stated that inner
speech is nonarticulatory and nonauditory and that ``Even the lowest
level units for inner speech are highly abstract'' (p.122).

In contrast with this Abstraction view, the physicalist or embodied view
considers inner speech production as mental simulation of overt speech
production. As such, it can be viewed as similar to overt speech
production, except that the motor execution process is blocked and no
sound is produced (Grèzes \& Decety, 2001; Postma \& Noordanus, 1996).
Under this Motor Simulation view, a continuum exists between overt and
covert speech, in line with the continuum drawn by Decety and Jeannerod
(1996) between imagined and actual actions. This hypoth- esis has led
certain authors to claim that inner speech by essence should share
features with speech motor actions (Feinberg, 1978; Jones \& Fernyhough,
2007). The Motor Simulation view is supported by several findings.
Firstly, covert and overt speech have comparable physiological
correlates: for instance, measurements of speaking rate (Landauer, 1962;
Netsell, Ashley, \& Bakker, 2010) and respiratory rate (Conrad \&
Schönle, 1979) are similar in both. A prediction of the Motor Simulation
view is that the speech motor system should be recruited during inner
speech. Subtle muscle activity has been detected in the speech
musculature using electromyography (EMG) during verbal mental imagery,
silent reading, silent recitation (Jacobson, 1931; Sokolov, 1972;
Livesay, Liebke, Samaras, \& Stanley, 1996; McGuigan \& Dollins, 1989),
and during auditory verbal hallucination in patients with schizophrenia
(Rapin, Dohen, Polosan, Perrier, \& L \& venbruck, 2013). Secondly, it
has been shown that covert speech production involves a similar cerebral
network as that of overt speech production. Covert and overt speech both
recruit essential language areas in the left hemisphere (for a review,
see Perrone- Bertolotti et al., 2014). However, there are differences.
Consistent with the Motor Simulation view and the notion of a continuum
between covert and overt speech, overt speech is associated with more
activity in motor and premotor areas than inner speech (e.g., Palmer et
al., 2001). This can be related to the absence of articulatory movements
during inner verbal production. In a reciprocal way, inner speech
involves cerebral areas that are not activated during overt speech
(Basho, Palmer, Rubio, Wulfeck, \& Müller, 2007). Some of these
activations (cingulate gyrus and superior rostral frontal cortex) can be
attributed to the inhibition of overt responses.

These findings suggest that the processes involved in overt speech
include those required for inner speech (except for inhibition). Several
studies in patients with aphasia support this view: overt speech loss
can either be associated with an impairment in inner speech (e.g.,
Levine, Calvanio, \& Popovics, 1982; Martin \& Caramazza, 1982) or with
intact inner speech: only the later phases of speech production
(execution) being affected by the lesion (Baddeley \& Wilson, 1985;
Marshall et al., 1985; Vallar \& Cappa, 1987). Geva, Bennett, Warburton,
and Patterson (2011) have reported a dissociation that goes against this
view, however. In three patients with chronic post-stroke aphasia (out
of 27 patients), poorer homophone and rhyme judgement performance was in
fact observed in covert mode compared with overt mode. A limitation of
this study, though, was that the task was to detect rhymes in written
words, which could have been too difficult for the patients. To over-
come this limitation, Langland-Hassan, Faries, Richardson, and Dietz
(2015) have tested aphasia patients with a similar task, using images
rather than written words. They also found that most patients performed
better in the overt than in the covert mode. They inferred from these
results that inner speech might be more demanding in terms of cognitive
and linguistic load, and that inner speech may be a distinct ability,
with its own neural substrates. We suggest an alternative interpretation
to this dissociation. According to our view, rhyme and homophone
judgements rely on auditory representations of the stimuli (see e.g.,
Paulesu, Frith, \& Frackowiak, 1993). Overt speech provides a strong
acoustic output that is fed back to the auditory cortex and can create
an auditory trace, which can be used to monitor speech. In the covert
mode, the auditory output is only mentally simulated, and its saliency
in the auditory system is lesser than in the overt mode. This is in
accordance with the finding that inner speech is associated with reduced
sensory cortex activation compared with overt speech (Shuster \&
Lemieux, 2005). In patients with aphasia, the weakened saliency of
covert auditory signals may be accentuated for two reasons: first,
because of impairment in the motor-to-auditory transformation that
produces the auditory simulation, and second, because of asso- ciated
auditory deficits. Therefore, according to our view, the reduced
performance observed in rhyme and homophone judgement tasks in the
covert compared with the overt mode in brain-injured patients, simply
indicates a lower saliency of the auditory sensations evoked during
inner speech compared with the actual auditory sensations fed back
during overt speech production. In summary, these findings suggest that
overt and covert speech share common subjective, physiological and
neural correlates, supporting the claim that inner speech is a motor
simulation of overt speech.

However, the Motor Simulation view has been challenged by several
experimental results. Examining the properties of errors during the
production of tongue twisters, Oppenheim and Dell (2010) showed that
speech errors display a lexical bias in both overt and inner speech.
According to these researchers, errors also display a phonemic similar-
ity effect (or articulatory bias), a tendency to exchange phonemes with
common articulatory features, but this second effect is only observed
with overt speech or with inner speech accompanied with mouthing. This
has led Oppenheim and Dell (2010) to claim that inner speech is fully
specified at the lexical level, but that it is impoverished at lower
featural (articulatory) levels. This claim, related to the Abstraction
view, is still debated however, as a phonemic similarity effect has been
found by Corley, Brocklehurst and Moat (2011). Their findings suggest
that inner speech is in fact specified at the articulatory level, even
when there is no intention to articulate words overtly. Other findings
however, may still challenge the Motor Simulation view. Netsell et al.
(2010) have examined covert and overt speech in persons who stutter
(PWS) and typical speakers. They have found that PWS were faster in
covert than in overt speech while typical speakers presented similar
overt and covert speech rates. This can be interpreted in favour of the
Abstraction view, in which inner representations are not fully specified
at the articulatory level, which would explain why they are not
disrupted in PWS speech. Altogether, these results suggest that full
articulatory specification may not always be necessary for inner speech
to be produced.

The aim of this study is to examine the physiological correlates of
verbal rumination in an attempt to provide new data in the debate
between motor simulation and abstraction. A prediction of the Motor
Simulation view is that verbal rumination, as a kind of inner speech,
should be accompanied with activity in speech-related facial muscles, as
well as in negative emotion or anxiety-related facial muscles, but
should not involve non-facial muscles (such as arm muscles).
Alternatively, the Abstraction view predicts that verbal rumination
should be associated with an increase in emotion-related facial
activity, without activity in speech-related muscles and non-facial
muscles.

There is strong interest in the examination of physiological correlates
of rumination as traditional assessment of rumination essentially
consists of self-reported measures. The measurement of rumination as
conceptualized by Nolen-Hoeksema (1991) was operationalized by the
development of the Ruminative Response Scale (RRS), which is a subscale
of the response style questionnaire (Nolen-Hoeksema \& Morrow, 1991).
The RRS consists of 22 items that describe responses to dysphoric mood
that are self-focused, symptom-focused, and focused on the causes and
consequences of one's mood. Based on this scale, Treynor, Gonzalez and
Nolen-Hoeksema (2003) have offered a detailed description of rumina-
tion styles and more recently, Watkins (2004, 2008) has further
characterized different modes of rumination. The validity of these
descriptions is nevertheless based on the hypothesis that individuals
have direct and reliable access to their internal states. However, self-
reports increase reconstruction biases (e.g., Brewer, 1986; Conway,
1990) and it is well known that participants have a very low level of
awareness of the cognitive processes that underlie and modulate complex
behaviors (Nisbett \& Wilson, 1977).

In order to overcome these difficulties, some authors have at- tempted
to quantify state rumination and trait rumination more objectively, by
recording physiological or neuroanatomical correlates of rumination (for
a review, see Siegle \& Thayer, 2003). Peripheral physiological
manifestations (e.g., pupil dilation, blood pressure, cardiac rhythm,
cardiac variability) have been examined during induced or chronic
rumination. Vickers and Vogeltanz-Holm (2003) have observed an increase
in systolic blood pressure after rumination induction, suggesting the
involvement of the autonomic nervous system in rumination. Moreover,
galvanic skin response has shown to be increased after a rumination
induction, in highly anxious women (Sigmon, Dorhofer, Rohan, \& Boulard,
2000). According to Siegle and Thayer (2003), disrupted autonomic
activity could provide a reliable physiological correlate of rumination.
In this line, Key, Campbell, Bacon, and Gerin (2008) have observed a
diminution of the high- frequency component of heart rate variability
(HF-HRV) after rumina- tion induction in people with a low tendency to
ruminate (see also Woody, McGeary, \& Gibb, 2014). A consistent link
between persevera- tive cognition and decreased HRV was also found in a
meta-analysis conducted by Ottaviani et al. (2015). Based on these
positive results and on suggestions that labial EMG activity may
accompany inner speech and therefore rumination, our aim was to examine
facial EMG as a potential correlate of rumination and HRV as an index to
examine concurrent validity.

In addition to labial muscular activity, we also recorded forehead
muscular activity (i.e., frontalis muscle) because of its implication in
prototypical expression of sadness (e.g., Ekman, 2003; Kohler et al.,
2004), reactions to unpleasant stimuli (Jäncke, Vogt, Musial, Lutz, \&
Kalveram, 1996), and anxiety or negative emotional state (Conrad \&
Roth, 2007)\footnote{The corrugator supercilii was another potential
  site, as it is sensitive to negative emotions. However, it has been
  claimed to be mostly activated for strong emotions such as
  fear/terror, anger/rage and sadness/grief (Ekman \& Friesen, 1978;
  Sumitsuji, Matsumoto, Tanaka, Kashiwagi, \& Kaneko, 1967). The
  rumination induction used in this study was designed to have
  participants self-reflect and brood over their failure at the IQ-
  test. It was not meant to induce such strong emotions. Several studies
  have reported increased activity in the frontalis muscle at rest in
  anxious or generalized anxiety disorder patients (for a review see
  Conrad \& Roth, 2007). We expected the type of emotional state induced
  by rumination to be closer to anxiety or worry than to strong emotions
  like fear, anger or grief. It was therefore more appropriate to record
  non-speech facial activity in the frontalis rather than in the
  corrugator.}. Our hypothesis was that frontalis activity could be an
accurate electromyographic correlate of induced rumination, as a
negatively valenced mental process.

In this study, we were also interested in the effects of relaxation on
induced rumination. Using a relaxation procedure targeted on muscles
involved in speech production is a further way to test the reciprocity
of the link between inner speech (verbal rumination) and orofacial
muscle activity. If verbal rumination is a kind of action, then its
production should be modulated in return by the effects of relaxation on
speech effectors. This idea is supported by the results of (among
others) Cefidekhanie, Savariaux, Sato and Schwartz (2014), who have
observed substantial perturbations of inner speech production while
participants had to realize forced movements of the articulators.

In summary, the current study aimed at evaluating the Motor Simulation
view and the Abstraction view by using objective and subjective measures
of verbal rumination. To test the involvement of the orofacial motor
system in verbal rumination, we used two basic approaches. In the first
approach, we induced verbal rumination and examined concurrent changes
in facial muscle activity (Experiment 1). In the second approach, we
examined whether orofacial relaxation would reduce verbal rumination
levels (Experiment 2). More specifi- cally, in Experiment 1, we aimed to
provide an objective assessment of verbal rumination using quantitative
physiological measures. Thus, we used EMG recordings of muscle activity
during rumination, focusing on the comparison of speech-related (i.e.,
two lip muscles − orbicularis oris superior and orbicularis oris
inferior) and speech-unrelated (i.e., forehead −frontalis- and forearm −
flexor carpi radialis) muscles. Under the Motor Simulation view, an
increase in lip and forehead EMG activity should be observed after
rumination induction, with no change in forearm EMG activity, associated
with an increase in self-reported rumination. Alternatively, under the
Abstraction view, an increase in forehead activity should be observed,
associated with an increase in self-reported rumination, and no changes
in either lip or forearm activity should be noted.

In Experiment 2, in order to assess the reciprocity of the rumination
and orofacial motor activity relationship, we evaluated the effects of
orofacial relaxation on rumination. More specifically, we compared three
kinds of relaxation: i) Orofacial Relaxation (i.e., lip muscles), ii)
Arm Relaxation (i.e., to differentiate effects specific to
speech-related muscle relaxation) and iii) Story Relaxation (i.e., to
differentiate effects specific to attentional distraction). If the Motor
simulation view is correct, we predicted a larger decrease of lip and
forehead muscle activity after an Orofacial Relaxation than after an Arm
Relaxation (associated with a larger decrease in self-reported
rumination), which should also be larger than after listening to a
story. We also predicted that forearm activity should remain stable
across the three conditions (i.e., should not decrease after
relaxation). Alternatively, if the Abstraction view is correct, we
predicted that none of the relaxation conditions should have an effect
on lip or arm activity, because none of these should have increased
after induction. However, we expected to observe a decrease in forehead
activity and self-reported rumination after Orofacial or Arm relaxation,
this decrease being larger than after listening to a Story. Importantly,
we predicted that, under the Abstraction View no superiority of the
Orofacial relaxation should be observed over the Arm relaxation.

\section{Methods}\label{methods}

\subsection{Participants}\label{participants}

Because of the higher prevalence of rumination in women than in men (see
Johnson \& Whisman, 2013; for a recent meta-analysis), we chose to
include female participants only. Seventy-two female under- graduate
students from Université Grenoble Alpes, native French speaking,
participated in our study. One participant presenting aberrant data
(probably due to inadequate sensor sticking) was removed from analyses.
Final sample consisted of seventy-one undergraduate female students
(Mage = 20.58, SDage = 4.99). They were recruited by e-mail diffusion
lists and participated in the experiment for course credits. They did
not know the goals of the study. The cover story presented the research
as aiming at validating a new I.Q. test, more sensitive to personality
profiles. Participants reported having no neurologic or psychiatric
medical history, no language disorder, no hearing deficit, and taking no
medication. Each participant gave written consent and this study has
been approved by the local ethical committee (CERNI, N° 2015-03-03-61).

\subsection{Material}\label{material}

\ldots{}

EMG signals were detected with TrignoTM Mini sensors (Delsys Inc.) at a
sampling rate of 1926 samples/s with a band pass of 20 Hz (12 dB/ oct)
to 450 Hz (24 dB/oct) and were amplified by a TrignoTM 16-channel
wireless EMG system (Delsys Inc.). The sensors consisted of two 5 mm
long, 1 mm wide parallel bars, spaced by 10 mm, which were attached to
the skin using double-sided adhesive interfaces. The skin was cleaned by
gently scrubbing it with 70\% isopropynol alcohol. EMG signals were then
synchronized using the PowerLab 16/35 (ADInstrument, PL3516). Raw data
from the EMG sensors were then resampled at a rate of 1 kHz and stored
in digital format using Labchart 8 software (ADInstrument, MLU60/8). As
shown in Fig. 1, bipolar surface EMG recordings were obtained from two
speech-related labial muscles: orbicularis oris superior (OOS) and
orbicularis oris inferior (OOI), as well as from one non speech- related
but negative-affect-related facial muscle: frontalis (FRO) and from one
non-facial and non speech-related muscle: flexor carpi radialis (FCR) on
the non-dominant forearm. The latter pair of electrodes was used to
check whether the rumination induction would cause any muscle
contraction, outside of the facial muscles. The same sensor layout was
used for all participants. Asymmetrical movements of the face have been
shown in speech and emotional expression. As reviewed in Everdell,
Marsh, Yurick, Munhall, and Paré (2007), the dominant side of the face
displays larger movements than the left during speech production,
whereas the non-dominant side is more emotionally expressive. To
optimise the capture of speech-related activity, the OOS and OOI sensors
were therefore positioned on the dominant side of the body (i.e.~the
right side for right-handed participants). To optimise the capture of
emotion-related activity, the FRO sensor was positioned on the
non-dominant side. To minimise the presence of involuntary manual
gestures during the recording, the FCR sensor was positioned on the
non-dominant side. Each pair of electrodes was placed parallel with the
direction of the muscle fibers, at a position distant from the
innervation zones and the muscle tendon interface, following the
recommendations of DeLuca (1997). The experiment was video-monitored
using a Sony HDR-CX240E video camera to track any visible facial
movements. A microphone was placed 20--30 cm away from the participant's
lips to record any faint vocal production during rumination. Stimuli
were displayed with E-prime 2.0 (\url{http://www}. pstnet.com) on a
19-inch color monitor.

\begin{figure}

{\centering \includegraphics[width=1\linewidth]{assets/face_emg} 

}

\caption{Facial muscles of interest. Two speech-related labial muscles: orbicularis oris superior (OOS) and orbicularis oris inferior (OOI); as well as one non speech-related but sadness-related facial muscle: frontalis (Front).}\label{fig:unnamed-chunk-1}
\end{figure}

\subsection{Procedure}\label{procedure}

This study consisted of two parts. The first part was carried out a week
before the EMG experiment and consisted in checking the inclusion
criteria. We checked that participants did not exceed a threshold on a
depressive symptoms scale. This was assessed using the French version of
the Center for Epidemiologic Studies Depression scale (CES-D; Fuhrer \&
Rouillon, 1989), which evaluates the level of depres- sive symptom in
subclinical population. We also collected information about any
potential speech, neurologic, neuromuscular or cardiac disorders and
about academic curriculum. Finally, the tendency to ruminate (i.e.,
trait rumination) in daily life was evaluated using the French version
of the Mini-CERTS (Cambridge-Exeter Repetitive Thought Scale; Douilliez,
Philippot, Heeren, Watkins, \& Barnard, 2014). The second part included
two EMG interdependent experiments related to Rumination Induction and
Rumination Reduction by Muscle Relaxation. Specifically, Experiment 1
consisted of acquiring physiolo- gical EMG data during rest and induced
rumination and Experiment 2 consisted of acquiring physiological EMG
data after different kinds of relaxation (see below).

During both Experiment 1 and Experiment 2, momentary rumination was
assessed using four different Visual Analogue Scales (VAS, the first two
being adapted and translated to French from Huffziger, Ebner- Priemer,
Koudela, Reinhard, \& Kuehner, 2012) rated from 0 to 100: i) ``At this
moment, I am thinking about my feelings'' (referred to as VAS
``Feelings''), ii) ``At this moment, I am thinking about my problems''
(referred to as VAS ``Problems''), iii) ``At this moment, I am brooding
about negative things'' (referred to as VAS ``Brooding'') and iv) ``At
this moment, I am focused on myself'' (referred to as VAS ``Focused'').

\subsubsection{Experiment 1: rumination
induction}\label{experiment-1-rumination-induction}

Participants were seated in front of a computer screen in a comfortable
and quiet room. EMG sensors were positioned as explained above (see Fig.
1). Before the rumination induction, each participant underwent a
non-specific relaxation session (i.e., without targeting specific
muscles) in order to minimize inter-individual initial thymic
variability (approximate duration ∼330 s). Immediately after, partici-
pants were instructed to remain silent and not to move for one minute to
carry out EMG ``baseline'' measurements. Then, participants' initial
level of rumination was assessed using the four VASs.

Subsequently, participants were invited to perform a 15-min I.Q. test,
which was presented on the computer screen facing them. They were
instructed to correctly respond to three types of I.Q. questions
(logical, mathematical and spatial-reasoning questions) in a very short
time (30 s). Most of the questions were very difficult, if not
impossible, to correctly answer in 30 s. We included ten different
questions for each of the three types of IQ question: ten logical
questions (e.g., finding the next number of a Fibonacci sequence), ten
mathematical questions (e.g., ``What is the result of the following
calculus: (30/165) − (70/ 66)'') and ten spatial-reasoning questions
(e.g., finding the next figure of a series). Forced-failure tasks have
extensively been employed in the literature to induce a slightly
negative mood, ideal for subsequent rumination induction (e.g., LeMoult
\& Joormann, 2014; Van Randenborgh, Hüffmeier, LeMoult, \& Joormann,
2010).

After the I.Q. test, participants were invited to reflect upon the
causes and consequences of their feelings, during five minutes (rumina-
tion induction). This method is based on the induction paradigm
developed by Nolen-Hoeksema and Morrow (1993). The classical paradigm
uses a series of prompts. In order to avoid the potential confound in
muscle activity induced by silent reading, we did not use the full
paradigm. We simply summarised the series of prompts by one typical
induction sentence. During this period, participants were asked to
remain silent and not to move, while EMG recordings were carried out
(i.e., EMG Post-induction measures). EMG signals of rumination were
collected during the last minute of this period. Finally, partici- pants
were instructed to self-report momentary rumination on the four VASs.

\subsubsection{Experiment 2: rumination reduction by
relaxation}\label{experiment-2-rumination-reduction-by-relaxation}

After Experiment 1, participants were randomly allocated to one of three
groups. In the first group, participants listened to a pre-recorded
relaxation session that was focused on orofacial speech-related muscles
(``Orofacial Relaxation'' condition). In the second group, relaxation
was focused on the arm muscles (``Arm Relaxation'' condition). In the
third group, participants simply listened to a story, read by the same
person, for an equivalent duration (``Story'' condition, detailed
content of the story can be found in the Supplementary Materials, in
French). In summary, the first condition allowed us to evaluate the
effects of targeted speech muscle relaxation on rumination. The second
condition allowed evaluating the effects of a non-orofacial relaxation
(i.e., speech- unrelated muscles) while the third condition allowed
controlling for effects of attentional distraction during relaxation
listening.

The speeches associated with the three conditions, relaxation sessions
and story listening session, were delivered to the participants through
loudspeakers. They were recorded by a professional sophrology therapist
in an anechoic room at GIPSA-lab (Grenoble, France) and were
approximately of the same duration (around 330 s).

After the relaxation/distraction session, participants were asked to
remain silent and not to move during one minute, during which EMG
measurements were collected (EMG Post-relaxation measures). Finally,
participants were instructed to self-report rumination on the four VASs.

\subsection{Data processing and
analysis}\label{data-processing-and-analysis}

\subsubsection{EMG data processing}\label{emg-data-processing}

EMG signal pre-processing was carried out using Labchart 8. The EMG data
were high-pass filtered using a Finite Impulse Response (FIR) filter at
a cut-off of 20 Hz, using the Kaiser window method with β = 6. Then,
output of this first filter was to a low-pass filtered at a cut-off of
450 Hz (with the same parameters), in order to focus on the 20--450 Hz
frequency band, following current recommendations for facial EMG studies
(DeLuca, 1997; DeLuca, Gilmore, Kuznetsov, \& Roy, 2010; Van Boxtel,
2001).

Although we specifically asked participants to remain silent and not to
move during EMG data collection, tiny facial movements (such as biting
one's lips) or vocal productions sometimes occurred. Periods with such
facial movement or vocal production were excluded from the analysis. To
do this, visual inspection of audio, video, and EMG signal was
performed. Specifically, for the EMG signals, we compared two methods of
signal selection. The first one consisted of setting a threshold on the
absolute value of the EMG signal and portions of signals above this
threshold were removed. This threshold was empiri- cally chosen using
visual inspection of a few samples and set to the mean EMG value plus 6
SDs. The second method consisted of manually removing periods of time
that included visually obvious bursts of EMG activity, corresponding to
overt contraction (as in Rapin et al., 2013). Based on samples from a
few participants, the comparisons between these two methods showed that
the automatic threshold method was somewhat less sensitive to overt
movements. Therefore, the second method was used, as it was more
conservative and less prone to leave data related to irrelevant overt
movements.

After pre-processing, EMG data were exported from Labchart soft- ware to
Matlab r2014a (Version 8.3.0.532, www.mathworks.fr). For each EMG
signal, mean values were computed under Matlab, using 200 ms sliding
windows. The average of these mean values were calculated for each
recording session (baseline, after induction and after
relaxation/induction). This provided a score for each muscle of interest
(OOS, OOI, FCR, FRO) in each Session (Baseline, Post- Induction,
Post-Relaxation) for each participant\footnote{Because of constraints
  attributable to the design of our experiment, we were not able to
  perform conventional control measures (e.g., time of the day, food
  consumption, sport activity, smoking habits, etc.). Moreover, in our
  study, periods of signal recording had to be shorter than usual HRV
  analysis time periods (cf.~methodology section). Although recent
  studies suggest that ``ultrashort term'' HRV analysis seems to
  correlate quite well with HRV analysis performed on longer periods of
  time (Brisinda et al., 2013; Salahuddin, Cho, Gi Jeong,\&Kim, 2007),
  we cannot exclude that our measurements might be unreliable. For these
  reasons, we chose not to present HRV results in this report and to
  focus on EMG results as well as subjective reports of rumination.}.

\subsubsection{Statistical analyses}\label{statistical-analyses}

Absolute EMG values are not meaningful as muscle activation is never
null, even in resting conditions, due in part to physiological noise
(Tassinary, Cacioppo, \& Vanman, 2007). In addition, there are inter-
individual variations in the amount of EMG activity in the baseline. To
normalise for baseline activity across participants, we used a
differential measure and expressed EMG amplitude as a percentage of
baseline level (Experiment 1) or of post-induction level (Experiment 2).

To model EMG amplitude variations in response to the rumination
induction (Experiment 1) and relaxation (Experiment 2), we used a
bayesian multivariate regression model with the natural logarithm of the
EMG amplitude (expressed in\% of baseline level) as an outcome, in an
intercept-only model (in Experiment 1), and using Condition (Orofacial,
Arm or Story) as a categorical predictor in Experiment 2. We used the
same strategy (two multivariate models) to analyse VAS scores (expressed
in relative changes) along the two experiments.

These analyses were conducted using RStudio (RStudio Team, 2015) and the
brms package (Bürkner, in press), an R implementation of Bayesian
multilevel models that employs the probabilistic program- ming language,
Stan (Carpenter et al., 2016). Stan implements gradient- based Markov
Chain Monte Carlo (MCMC) algorithms (e.g., Hamilto- nian Monte-Carlo),
which allow yielding posterior distributions that are straightforward to
use for interval estimation around all parameters. Two MCMC simulations
(or ``chains'') were run for each model, including 100,000 iterations, a
warmup of 10,000 iterations, and a thinning interval of 10. Posterior
convergence was assessed examining autocorrelation and trace plots, as
well as the Gelman-Rubin statistic. Fixed effects were estimated via the
posterior mean and 95\% highest density intervals (HDIs), where an HDI
interval is the Bayesian analogue of a classical confidence
interval\footnote{While not suffering from the misunderstandings
  associated with frequentist confidence intervals (for more details,
  see for instance Morey, Hoekstra, Rouder, Lee \& Wagenmakers, 2015).}.

This strategy allowed us to examine posterior probability distribution
on each parameter of interest (i.e., effects of session and condition on
each response variable). When applicable, we also report evidence ratios
(ERs), computed using the hypothesis function of the brms package
(Bürkner, in press). These evidence ratios are simply the posterior
probability under a hypothesis against its alternative (Bürkner, in
press). We also report summary statistics (mean and HDI) of Cohen's d
effect sizes, computed from the posterior samples.

\section{Results}\label{results}

\subsection{Experiment 1: rumination
induction}\label{experiment-1-rumination-induction-1}

The evolution of VAS scores (for the four assessed scales: Feelings,
Problems, Brooding, and Focused) and EMG (for the four muscles: OOS,
OOI, FCR and FRO) activity from baseline to post-induction were
examined.

\subsubsection{Self-reported rumination measures: VAS
scores}\label{self-reported-rumination-measures-vas-scores}

Results for VAS relative changes based on the multivariate models
described earlier are shown in the right panel of Fig. 2. Thereafter,
\(\alpha\) represents the mean of the posterior distribution of the
intercept. Raw pre- and post-induction scores are provided in
Supplementary Materials.

Mean VAS score on the Feelings scale was slightly lower after induction
(\(\alpha\) = −5.55, 95\% HDI {[}-10.89, −0.24{]}, d = −0.23, 95\% HDI
{[}-0.46, −0.01{]}), while Problems score was slightly higher
(\(\alpha\) = 3.99, 95\% HDI {[}-2.04, 9.83{]}, d = 0.15, 95\% HDI
{[}-0.08, 0.37{]}). We observed a strong increase of the score on the
Brooding scale (\(\alpha\) = 14.45, 95\% HDI {[}8.07, 20.72{]}, d =
0.50, 95\% HDI {[}0.26, 0.74{]}), and a strong decrease on the Focused
scale (\(\alpha\) = −11.63, 95\% HDI {[}-17, −6.07{]}, d = −0.48, 95\%
HDI {[}-0.72, −0.24{]}). As we examined the fit of the intercept-only
model, these estimates represent the posterior mean for each muscle.

In the following, we report the mean (indicated by the Greek symbol ρ)
and the 95\% HDI of the posterior distribution on the correlation
coefficient (\(\rho\)). Examination of the correlation matrix estimated
by the multivariate model revealed no apparent correlation neither
between Feelings and Problems scales (\(\rho\) = −0.01, 95\% HDI
{[}-0.23, 0.22{]}), nor between Feelings and Brooding (\(\rho\) = 0.08,
95\% HDI {[}-0.15, 0.30{]}). However, we observed a strong positive
correlation between Problems and Brooding VASs (\(\rho\) = 0.64, 95\%
HDI {[}.49, 0.76{]}), a positive correlation between Feelings and
Focused (\(\rho\) = 0.30, 95\% HDI {[}.08, 0.50{]}), and a negative
correlation between Problems and Focused (\(\rho\) = −0.30, 95\% HDI
{[}-0.49, −0.08{]}), as well as between Brooding and Focused (\(\rho\) =
−0.18, 95\% HDI {[}-0.39, 0.05{]}).

\begin{figure}

{\centering \includegraphics[width=1\linewidth]{assets/emg_fig1} 

}

\caption{Posterior mean (white dots) and 95\% credible intervals for the EMG amplitude (expressed in percentage of baseline level, left panel), and the VAS score (expressed in relative change from baseline, right panel). N = 71 (for each muscle and each VAS). Dashed line represents the null value (i.e., 100).}\label{fig:unnamed-chunk-2}
\end{figure}

\subsubsection{EMG}\label{emg}

Results for EMG data based on the multivariate model described earlier
are shown in the left panel of Figure 2. Summary statistics were
computed on posterior samples transformed back from log scale.

Mean EMG amplitude for OOS was higher after induction (\(\alpha\) =
138.57, 95\% HDI {[}124.43, 151.71{]}, d = 0.66, 95\% HDI {[}0.49,
0.84{]}) as well as for OOI (\(\alpha\) = 163.89, 95\% HDI {[}145.24,
184.14{]}, d = 0.77, 95\% HDI {[}0.61, 0.94{]}), and FRO (\(\alpha\) =
197.55, 95\% HDI {[}166.59, 228.42{]}, d = 0.74, 95\% HDI {[}0.59,
0.89{]}). Effects on the FCR were approximately null (\(\alpha\) =
100.10, 95\% HDI {[}97.48, 102.76{]}, d = 0.01, 95\% HDI {[}-0.24,
0.23{]}).

Examination of the correlation matrix estimated by the bayesian
multivariate model revealed a positive correlation between OOS and OOI
EMG amplitudes (\(\rho\) = 0.44, 95\% HDI {[}.24, 0.61{]}), while no
apparent correlations neither between OOS and FCR (\(\rho\) = 0.09, 95\%
HDI {[}-0.14, 0.31{]}), OOS and FRO (\(\rho\) = 0.12, 95\% HDI {[}-0.11,
0.35{]}), OOI and FCR (\(\rho\) = 0.02, 95\% HDI {[}-0.21, 0.25{]}), FRO
and FCR (\(\rho\) = −0.06, 95\% HDI {[}-0.28, 0.17{]}), nor OOI and FRO
(\(\rho\) = 0.07, 95\% HDI {[}-0.16, 0.29{]}). Scatterplots, marginal
posterior distributions and posterior distributions on correlation
coefficients are available in Supplementary Materials (Supplementary
materials, data, reproducible code and figures are available at:
\url{https://osf.io/882te/}).

In order to check whether the propensity to ruminate could predict the
effects of the rumination induction on EMG amplitude, we compared the
multivariate model described above, with a similar model but with the
score on the abstract dimension of the Mini- CERTS as an additional
predictor. We compared these models using the widely applicable
information criterion (WAIC; Watanabe, 2010), via the WAIC function of
the brms package (Bürkner, in press). Results showed that the
intercept-only model had a lower WAIC (WAIC = 177.39) than the more
complex model (WAIC = 182.01), indicating that there is no predictive
benefit in adding the Mini-CERTS score as a predictor.

\subsubsection{Correlations between EMG amplitudes and VAS
scores}\label{correlations-between-emg-amplitudes-and-vas-scores}

Correlations between EMG amplitudes and VAS scores were exam- ined using
the BayesianFirstAid package (Bååth, 2013), using 15,000 iterations for
each correlation coefficient. Both estimated correlation coefficients
(\(\rho\)s) and 95\% HDIs are reported in Table 1.

\subsection{Experiment 2: rumination reduction by
relaxation}\label{experiment-2-rumination-reduction-by-relaxation-1}

In the second experiment, we aimed at comparing the evolution in EMG
activity and VAS scores from post-induction to post-relaxation in three
different conditions: Orofacial relaxation, Arm relaxation, and
listening to a Story.

\subsubsection{Self-reported rumination measures: VAS
scores}\label{self-reported-rumination-measures-vas-scores-1}

Posterior means and 95\% HDIs of the VAS scores in each condition of
experiment 2 are represented in Fig. 3 and Table 1 (Table 2).

In order to compare the effects of the two kind of relaxation on the VAS
scores, we then used the hypothesis function of the brms package that
allows deriving evidence ratios (ER). These evidence ratios are simply
the posterior probability under a hypothesis (e.g., the hypothesis that
the Orofacial relaxation session would be more effective in reducing
self-reported rumination than the Arm relaxation session) against its
alternative (Bürkner, in press).

Since the Problems and the Brooding scales seemed to be sensitive
markers of rumination (as their scores increased after induction in
Experiment 1), our analyses were focused on these two scales.

Concerning the Problems VAS, the decrease observed in the Orofacial
condition was more pronounced than in the Arm condition (Est = −11.06,
SE = 6.35, ER10 = 22.65), and slightly more pro- nounced compared to the
Story condition (Est = −6.05, SE = 6.31, ER10 = 4.98). The observed on
the Brooding VAS score in the Orofacial condition was larger than in the
Arm condition (Est = −9.98, SE = 6.07, ER10 = 18.85), and slightly more
important compared to the Story condition (Est = −5.23, SE = 6.01, ER10
= 4.27).

\begin{figure}

{\centering \includegraphics[width=1\linewidth]{assets/emg_fig2} 

}

\caption{Posterior mean and 95\% credible intervals for the VAS score (expressed in relative change from post-induction level).}\label{fig:unnamed-chunk-3}
\end{figure}

\subsubsection{EMG}\label{emg-1}

Posterior means and 95\% HDIs of the EMG amplitude in each condition of
experiment 2 are represented in Figure XX and reported in Table XX.

We used the same strategy as before to compare the effects of the two
kinds of relaxation on the EMG amplitudes.

Concerning the OOS, the observed decrease in the Orofacial condition was
more pronounced than in the Arm condition (Est = −0.34, SE = 0.14, ER10
= 140.73), as well as concerning the OOI (Est = −0.35, SE = 0.19, ER10 =
29.46), while we observed no noticeable differences between the two
kinds of relaxation concerning the EMG amplitude of the FRO (Est =
-0.04, SE = 0.14, ER10 = 1.53).

\begin{figure}

{\centering \includegraphics[width=1\linewidth]{assets/emg_fig3} 

}

\caption{Posterior mean and 95\% credible intervals for the VAS score (expressed in relative change from post-induction level).}\label{fig:unnamed-chunk-4}
\end{figure}

\section{Discussion}\label{discussion}

\subsection{Experiment 1}\label{experiment-1}

In the first experiment, we examined electromyographic correlates of
induced rumination in healthy individuals. According to the Motor
Simulation view, we predicted an increase in the activity of all facial
muscles after the rumination induction, associated with an increase in
self-reported rumination. Alternatively, the Abstraction view predicted
an increase in self-reported rumination associated with an increase in
forehead activity with no changes in either lip or forearm activity.

To test the predictions of these two theoretical views, we compared EMG
measures and VAS scores after induction to their values before
induction. EMG activity was examined in four muscles: OOS and OOI, two
muscles involved in speech production, FRO, a facial negative-
affect-related but not speech-related muscle, and FCR, a non-facial
control muscle on the non-dominant forearm.

As predicted by the Motor Simulation view, we observed an increase in
the activity of the two speech-related muscles (OOS \& OOI) as well as
in the negative-affect-related muscle (FRO) and no change in FCR
activity. The increase in facial EMG together with the increase in the
subjective reports of rumination suggests that facial EMG increase is a
correlate of verbal rumination. As supported by several studies results,
the forehead muscle activity has been associated with unpleasant
emotions (Jäncke et al., 1996) or anxiety (Conrad \& Roth, 2007). The
increase in FRO activity observed here is consistent with the increase
in negative emotions induced by our negatively valenced induction
procedure. Orbicularis oris lip muscles are associated with speech
production. The increase in lip activity observed here suggests that the
speech motor system was involved during the ruminative phase. The fact
that the FCR remained stable after rumination induction suggests that
the observed facial activity increase was not due to general body
tension induced by a negative mental state. These facial EMG results
therefore support the hypothesis that rumination is an instance of
articulatory-specified inner speech.

After the rumination induction, a larger increase in OOI activity was
observed compared to the increase in OOS activity. This finding is
consistent with previous findings of higher EMG amplitude in the lower
lip during speech and inner speech (e.g., Barlow \& Netsell, 1986;
Regalo et al., 2005; Sokolov, 1972) or auditory verbal hallucinations
(Rapin et al., 2013). Rapin et al. (2013) have explained the difference
between the activities of the two lip muscles by muscle anatomy. The
proximity of the OOI muscle with other speech muscles (such as the
depressor angular muscle or the mentalis) could increase the surface EMG
signal captured on the lower lip (OOI), as compared to the upper lip
(OOS) during speech. An even larger increase in FRO activity was
observed compared to the increase in lip muscle activity. As EMG
amplitude is known to vary with muscle length (Babault, Pousson,
Michaut, \& Van Hoecke, 2003), the greater increase in frontalis
activity could be explained by its anatomical properties.

However, although a functional distinction can be drawn between the
forehead and the lip muscles, one should acknowledge the fact that these
two sets of muscles can be commonly activated during some behaviours.
For instance, Van Boxtel \& Jessurun (1993) have shown that orbicularis
oris inferior and frontalis were both activated during a two-choice
serial reaction task in which nonverbal auditory or visual signals were
presented. Moreover, there was a gradual increase in EMG activity in
these muscles during the task, either when the task was prolonged or
when the task was made more difficult. They interpreted this increase in
EMG activity as associated with a growing compensa- tory effort to keep
performance at an adequate level. An alternative interpretation is that
the increase in task difficulty was dealt with by inner verbalization.
Covertly rehearsing the instructions or covertly qualifying the stimuli
might have helped the participants to perform adequately. Therefore, the
increase in orbicularis oris activity might have been related to an
increase in covert verbalization, whereas the increase in frontalis
activity might have been related to increased anxiety or tension. The
fact that the EMG increase was muscle specific, and that some facial
muscles (orbicularis oculi, zygomaticus major, temporalis) did not show
an increase in activity unless the task became too difficult, supports
this interpretation. It cannot be ruled out, however, that orbicularis
oris activity may in some cases be related to mental effort without
mental verbalisation. Nevertheless, although the IQ test itself was
designed to induce mental effort, no cognitively demanding task was
asked to the participant during the period of EMG recording (i.e.,
approximately four minutes after the end of the test). Although we
cannot absolutely exclude that rumination in itself could require
cognitive effort, it seems unlikely that mental effort was the main
factor of variation.

Scores on the VAS need to be discussed in further detail. We examined
which VAS scales were most suitable to capture changes in state
rumination to allow focused analyses. Due to the ``pre-baseline''
relaxation session, during which participants were asked to concentrate
on their body and breathing cycles, participants reported a high level
of attentional self-focus at baseline (``Feelings'' and ``Focused''
VAS). Because of the high level of self-focused attention at baseline,
it is likely that the scores on the ``Feelings'' and ``Focused'' VAS did
not show the expected increase after rumination induction (ceiling
effect). The scores on the scales ``Problems'' and ``Brooding'', which
are more representative of maladaptive rumination, did increase after
our rumination induction paradigm, however. Interestingly, the
``Brooding'' VAS corresponded to a larger increase and seemed to be more
sensitive to rumination induction than the ``Problems'' VAS. Given this
greater sensibility and the strong positive correlation between the
``Brooding'' and the ``Problems'' VAS, it thus make sense to consider
the ``Brooding'' VAS as a better estimate of ruminative state, at least
within our paradigm. We will therefore only use this scale to assess
rumination in the following.

The fact that we did not observe any association between the propensity
to ruminate (as measured by the Mini-CERTS questionnaire) and the
effects of the induction is consistent with the results of Rood,
Roelofs, Bögels, and Arntz (2012) who found that the level of trait
rumination did not moderate the effects of a rumination induction.

\subsection{Experiment 2}\label{experiment-2}

In the second experiment, we studied the effects of two muscle- specific
relaxation sessions: Orofacial relaxation and Arm relaxation. We
compared their effects to a third control condition (Story), which did
not involve the deliberate relaxation of any specific muscle. Our
predictions were that a decrease in facial EMG activity should be
observed in each condition. If the Motor Simulation view is correct, we
expected a larger decrease in the activity of all facial muscles in the
``Orofacial relaxation'' condition than in the ``Arm relaxation''
condition, associated with a larger decrease in self-reported
rumination. Additionally, we expected a more pronounced decrease in the
two relaxation conditions (orofacial and arm relaxation conditions) than
in the control (``Story'') condition. We also expected no difference
between relaxation conditions regarding the change in the forearm muscle
activity.

The data indicated a decrease in self-reported rumination (``Brooding''
VAS) in each condition. The ``Orofacial'' relaxation condi- tion
elicited a slightly larger decrease than the ``Arm relaxation'' or the
``Story'' condition. However, there was extensive individual variation
in response to these conditions. As concerns EMG results, we observed a
decrease in OOS and OOI activities in all three conditions but this
decrease was more pronounced in the orofacial condition than in the
other two conditions. The frontalis activity did not show the same
pattern. A similar FRO activity decrease was observed in both the
orofacial and the non-orofacial relaxation conditions. Therefore, in
Experiment 2, the lip muscles and the forehead muscle follow differ-
ential evolutions. A dissociation was observed: whereas both orofacial
and arm relaxations resulted in a decrease in forehead activity, only
orofacial relaxation was successful at reducing lip activity.

Considering both VAS results and the dissociation in EMG patterns,
several interpretations are possible. The first interpretation is that
verbal production associated with rumination was more reduced by
orofacial muscular relaxation than by non-orofacial relaxation. This
interpretation is consistent with the fact that the ``Brooding'' VAS was
slightly more decreased in this condition compared to the other two. The
larger decrease in OOS and OOI amplitude after orofacial relaxa- tion
would thus reflect this reduction in verbal production, as hypothesised
by the Motor Simulation view. The fact that FRO activity displayed a
similar decrease in both orofacial and non-orofacial relaxation
conditions could suggest that any means of body relaxation (be it
orofacial or not) is appropriate to reduce negative affect and can
therefore reduce forehead contraction. This suggests that the FRO
activity increase presumably reflected negative affect and tension (such
as observed in EMG studies on generalised anxiety disorder patients, see
Conrad \& Roth, 2007 for a review).

Alternatively, one could also argue that the larger decrease in lip
muscle activity after orofacial relaxation finds a more trivial explana-
tion in that it seems obvious to expect that orofacial relaxation will
be more efficient to reduce lip muscle contraction than non-orofacial
relaxation. Thus, the different impacts of the two relaxation sessions
on the lip muscles would not be related to reduced rumination per se but
simply to a more anatomically targeted relaxation. However, several
observations argue against such an interpretation. The larger decrease
in the ``Brooding'' VAS in the orofacial relaxation condition compared
with the other conditions suggests that the reduction in lip muscle
activity is indeed related to the reduction in rumination. Moreover, an
interpretation solely based on anatomical links does not explain why FRO
activity displayed the same amount of reduction in both relaxation
sessions. If reduction in muscle activity was merely related to the
effect of facial muscle relaxation, then the decrease in FRO activity
should have also been higher in the orofacial relaxation condition than
in the other relaxation condition, which was not the case. Therefore the
dissociation between forehead and lip patterns of activity, together
with the differential effects of the two types of relaxation on
subjective rumination reports strongly suggest that different processes
underlie the activity of these two sets of muscles. We therefore
consider that the first interpretation is more plausible: frontalis
activity seems related to overall facial tension due to negative affect
whereas lip activity seems to be related to the specific involvement of
the speech musculature in rumination. These results thus seem to confirm
the interpretation of decreased OOS and OOI activities in the orofacial
relaxation condition as markers of rumination reduction.

Interestingly, we observed no changes of forearm EMG activity in any of
the three conditions of experiment 2. The fact that the relaxation
session focused on the forearm was not associated with a decrease in FCR
activity has a simple explanation: FCR activity had not increased after
rumination induction and had remained at floor level. The forearm was
thus already relaxed and the Arm relaxation session did not modify FCR
activity. Another interesting conclusion related to this absence of
modification of forearm activity is that relaxation does not spuriously
decrease muscle activity below its resting level. One possible
interpretation of the increase in lip EMG after rumination induction
could have been that baseline relaxation artificially decreased baseline
activity under its resting level. The facts that forearm activity did
not decrease after arm-focused relaxation contradicts this
interpretation.

Finally, the ``Story'' condition was also associated with a decrease in
OOI and FRO activities. This could mean that listening to a story
reduced rumination to the same extent as relaxation did. However, the
discrepancy observed in ``Focused'' VAS between the two relaxation
conditions on the one hand and the control condition on the other hand,
suggests that the EMG decrease observed in the ``Story'' condition might
be attributable to a different cause than that observed in the two
relaxation conditions. Listening to a story could help reducing rumina-
tion by shifting attention away from ruminative thoughts. Relaxation
sessions could help reducing rumination by shifting attention to the
body in a beneficial way.

\subsection{General discussion}\label{general-discussion}

We set out two experiments to examine whether rumination involves motor
simulation or is better described as linguistically abstract and
articulatory impoverished. We used labial, facial, and arm EMG measures
to assess potential articulatory correlates of rumination. The patterns
of results of our study seem to be in favour of the motor nature of
verbal rumination. In Experiment 1, rumination induction was associated
with a higher score on the scale ``I am brooding about negative things''
which is representative of abstract- analytical rumination, considered
as verbal rumination. This maladap- tive rumination state was associated
with an increase in the activity of two speech-related muscles, without
modification of the arm muscle activity, which indicates that rumination
involves activity in speech articulatory muscles, specifically. The
concurrent increase in forehead muscle activity could be explained by an
increase in negative emotions induced by our negatively valenced
induction procedure. The results of Experiment 1 therefore show the
involvement of the speech muscula- ture during rumination. This is in
line with the Motor simulation view, according to which inner speech is
fully specified at the articulatory level, not just the lexical level.

In Experiment 2, guided relaxation resulted in a decrease in speech
muscle activity. In the lip muscles, the activity decrease was stronger
after orofacial relaxation than after arm-focused relaxation. In the
forehead muscle, however the effect was the same for both types of
relaxation. This decrease in speech muscle activity was associated with
a decrease in self-reports of rumination and was most pronounced after
orofacial relaxation. These findings suggest that a reduction in speech
muscle activity could hinder articulatory simulation and thus limit
inner speech production and therefore reduce rumination. This inter-
pretation is consistent with the Motor Simulation view of inner speech.
Brooding-type rumination was also diminished after the arm-focused
relaxation as well as after listening to a story, although less than in
the orofacial relaxation. This suggests that general relaxation or
distraction are also likely to reduce negative rumination. To summarize,
experi- ments 1 and 2 are consistent with the Motor Simulation view of
inner speech, according to which speech muscle activity is inherent to
inner speech production. Experiment 1 shows the involvement of the lip
musculature during brooding-type rumination. Experiment 2 suggests that
brooding-type rumination could be reduced by blocking or relaxing speech
muscles.

These data support the utility of labial EMG as a tool to objectively
assess inner speech in a variety of normal and pathological forms. We
suggest that this method could be used as a complement to self-report
measures, in order to overcome limitation of these measures.

Our results should be interpreted with some limitations in mind.
Firstly, our sample consisted exclusively of women. Although this
methodological choice makes sense considering the more frequent
occurrence of rumination in women, further studies should be con- ducted
to ascertain that our results may generalize to men. Secondly, in
Experiment 1, no between-subject control condition was used to compare
with the group of participants who underwent rumination induction. Thus,
we cannot rule out that other processes occurred between baseline and
rumination induction, influencing responding. Thirdly, substantial
inter-individual differences were observed concern- ing the size of the
effect of rumination induction on facial EMG activity. The results of
Jäncke (Jäncke, 1996; Jäncke et al., 1996) can shed light on this last
result. Jäncke used a similar procedure (i.e., negative mood induction
using a false I.Q. test and facial EMG measurements to assess emotions),
except that the experimenter was not in the room while participants
performed the test and acknowledged their results. The experimenter then
came back to the room and analysed participants' behaviours. Jäncke
observed an increase in facial muscular activity (assessed when
participants were reading their results) only in partici- pants who were
prone to express their distress when the experimenter came back, while
more introverted participants did not show any increased facial activity
when reading their results. Jäncke interpreted these results in the
framework of an ecological theory of facial expression, suggesting that
facial expressions would not only be guided by underlying emotions, but
also by their communicative properties. Considering these results, it
seems likely that the proneness of participants to communicate their
emotions could have mediated effects of the induction on their facial
EMG activity. This could partially explain the observed inter-individual
variability in facial EMG activity associated with rumination. Moreover,
even though rumination is a predominantly verbal process, one cannot
exclude that some of our participants experienced rumination in another
modality (e.g., ima- gery-based rumination), which would explain their
lower than average lip activity.

Thus, a logical next step is to examine qualitative factors that mediate
the link between rumination and facial muscular activity. These factors
(among others) could be proneness to communicate emotion or proneness to
verbalize affects. Additionally, recent studies suggest a link between
verbal aptitudes and propensity to ruminate. Uttl, Morin and Hamper
(2011) have observed a weak but consistent correlation between the
tendency to ruminate and scores on a verbal intelligence test. Penney,
Miedema and Mazmanian (2015) have observed that verbal intelligence
constitutes a unique predictor of rumination severity in chronic anxious
patients. To our knowledge, the link between verbal intelligence and
induced rumination has never been studied. It would be interesting to
examine whether the effects of a rumination induction could be mediated
by verbal intelligence, and to what extent this could influence related
facial EMG activity.

In conclusion, this study provides new evidence for the facial
embodiment of rumination, considered as a particular instance of inner
speech. Even if more data are needed to confirm these preliminary
conclusions, our results seem to support the Motor Simulation view of
inner speech production, manifested as verbal rumination. In addition,
facial EMG activity provides a useful means to objectively quantify the
presence of verbal rumination.

\section{Acknowledgements}\label{acknowledgements-1}

This project was funded by the ANR project INNERSPEECH {[}grant number
ANR-13-BSH2-0003-01{]}. The first author of the manuscript is funded by
a fellowship from Université Grenoble Alpes and a grant from the Pôle
Grenoble Cognition. We thank Nathalie Vallet for recording the
relaxation and distraction sessions. We thank our colleagues from
GIPSA-lab: Marion Dohen for her help in the recording of the audio
stimuli in the anechoic room at GIPSA-lab, as well as Christophe
Savariaux and Coriandre Vilain for their advice in the audio setup
associated with the EMG measures. We are also grateful to Rafael
Laboissière and Adeline Leclercq Samson for their advice concerning data
analysis. We sincerely thank two anonymous reviewers for their critical
reading of our manuscript and their many insightful comments and
suggestions. Access to the facility of the MSH-Alpes SCREEN platform for
conducting research is gratefully acknowledged.

\section{Supplementary data}\label{supplementary-data}

Supplementary data associated with this article can be found, in the
online version, at
\url{http://dx.doi.org/10.1016/j.biopsycho.2017.04.013}.

\begin{summary}{Summary of Chapter\getcurrentref{chapter}}

Blah blah ...

\end{summary}

\chapter{Dissociating facial electromyographic correlates of visual and
verbal induced
rumination}\label{dissociating-facial-electromyographic-correlates-of-visual-and-verbal-induced-rumination}

Summary of the research\ldots{}\footnote{This experimental chapter is a
  manuscript reformatted for the need of this thesis. Source: The
  manuscript has been submitted to Psychological Research.
  Pre-registered protocol, preprint, data, as well as reproducible code
  and figures are available at: \url{https://osf.io/c9pag/}.}

\begin{summary}{Summary of Chapter\getcurrentref{chapter}}

Blah blah ...

\end{summary}

\chapter{Muscle-specific electromyographic correlates of inner speech
production}\label{muscle-specific-electromyographic-correlates-of-inner-speech-production}

Summary of the research\ldots{}\footnote{This experimental chapter is a
  manuscript reformatted for the need of this thesis. Source: The
  manuscript has been submitted to Psychological Research.
  Pre-registered protocol, preprint, data, as well as reproducible code
  and figures are available at: \url{https://osf.io/czer4/}.}

\begin{summary}{Summary of Chapter\getcurrentref{chapter}}

Blah blah ...

\end{summary}

\chapter{Articulatory suppression effects on induced
rumination}\label{articulatory-suppression-effects-on-induced-rumination}

Summary of the research\ldots{}\footnote{This experimental chapter is a
  manuscript reformatted for the need of this thesis. Source: The
  manuscript has been submitted to Psychological Research.
  Pre-registered protocol, preprint, data, as well as reproducible code
  and figures are available at: \url{https://osf.io/3bh67/}.}

\begin{summary}{Summary of Chapter\getcurrentref{chapter}}

Blah blah ...

\end{summary}

\chapter{Refining the involvement of the speech motor system during
rumination: a dual-task
investigation}\label{refining-the-involvement-of-the-speech-motor-system-during-rumination-a-dual-task-investigation}

Summary of the research\ldots{}\footnote{This experimental chapter is a
  manuscript reformatted for the need of this thesis. Source: The
  manuscript has been submitted to Psychological Research.
  Pre-registered protocol, preprint, data, as well as reproducible code
  and figures are available at: \url{https://osf.io/8ab2d/}.}

\begin{summary}{Summary of Chapter\getcurrentref{chapter}}

Blah blah ...

\end{summary}

\part{Discussion}\label{part-discussion}

\chapter{Discussion and perspectives}\label{discussion-and-perspectives}

\section{Summary of the results}\label{summary-of-the-results}

\ldots{}

\section{Limitations and ways
forward}\label{limitations-and-ways-forward}

\ldots{}

\section{Conclusions}\label{conclusions}

\ldots{}

\noindent
\setlength{\parindent}{-0.20in} \setlength{\leftskip}{0.20in}
\setlength{\parskip}{8pt}

\bibliography{bib/thesis.bib,bib/packages.bib}


\end{document}
